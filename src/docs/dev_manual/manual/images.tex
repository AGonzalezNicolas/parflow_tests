\batchmode
\documentstyle[11pt,makeidx,parflow]{book}

\let\providecommand\newcommand
\RequirePackage{ifthen}





\setlength{\oddsidemargin}{0in} 

\setlength{\evensidemargin}{0in} 

\setlength{\textwidth}{6.5in} 

\setlength{\topmargin}{0in} 

\setlength{\textheight}{8.0in} 




% porosity
                  % density
         % solid density
                   % viscosity
                       % saturation
                % intrinsic permeability tensor
                    % relative permeability
                       % pressure
                       % source term
                       % pressure head
                % hydraulic conductivity tensor
                 % gravity constant
                % gravity vector
                % velocity vector
                % Darcy velocity vector
                % gauge pressure
              % scale factor
                % flux vector
                % unit normal
             % mobility


% contaminant
                       % mass concentraion


% injection well region
         % extraction well region


% injection well rate
 % extraction well rate


% degradation rate


% distribution coefficient


% total hydraulic conductivity
              % total mobility
                  % total Darcy velocity


% vertical 1/2





\makeindex








\usepackage[latin1]{inputenc}



\makeatletter
\AtBeginDocument{\makeatletter
\input /home/smithsg/parflow/pfclm_ms/src/docs/dev_manual/manual.aux
\makeatother
}

\makeatletter
\count@=\the\catcode`\_ \catcode`\_=8 
\newenvironment{tex2html_wrap}{}{}%
\catcode`\<=12\catcode`\_=\count@
\newcommand{\providedcommand}[1]{\expandafter\providecommand\csname #1\endcsname}%
\newcommand{\renewedcommand}[1]{\expandafter\providecommand\csname #1\endcsname{}%
  \expandafter\renewcommand\csname #1\endcsname}%
\newcommand{\newedenvironment}[1]{\newenvironment{#1}{}{}\renewenvironment{#1}}%
\let\newedcommand\renewedcommand
\let\renewedenvironment\newedenvironment
\makeatother
\let\mathon=$
\let\mathoff=$
\ifx\AtBeginDocument\undefined \newcommand{\AtBeginDocument}[1]{}\fi
\newbox\sizebox
\setlength{\hoffset}{0pt}\setlength{\voffset}{0pt}
\addtolength{\textheight}{\footskip}\setlength{\footskip}{0pt}
\addtolength{\textheight}{\topmargin}\setlength{\topmargin}{0pt}
\addtolength{\textheight}{\headheight}\setlength{\headheight}{0pt}
\addtolength{\textheight}{\headsep}\setlength{\headsep}{0pt}
\setlength{\textwidth}{349pt}
\newwrite\lthtmlwrite
\makeatletter
\let\realnormalsize=\normalsize
\global\topskip=2sp
\def\preveqno{}\let\real@float=\@float \let\realend@float=\end@float
\def\@float{\let\@savefreelist\@freelist\real@float}
\def\liih@math{\ifmmode$\else\bad@math\fi}
\def\end@float{\realend@float\global\let\@freelist\@savefreelist}
\let\real@dbflt=\@dbflt \let\end@dblfloat=\end@float
\let\@largefloatcheck=\relax
\let\if@boxedmulticols=\iftrue
\def\@dbflt{\let\@savefreelist\@freelist\real@dbflt}
\def\adjustnormalsize{\def\normalsize{\mathsurround=0pt \realnormalsize
 \parindent=0pt\abovedisplayskip=0pt\belowdisplayskip=0pt}%
 \def\phantompar{\csname par\endcsname}\normalsize}%
\def\lthtmltypeout#1{{\let\protect\string \immediate\write\lthtmlwrite{#1}}}%
\newcommand\lthtmlhboxmathA{\adjustnormalsize\setbox\sizebox=\hbox\bgroup\kern.05em }%
\newcommand\lthtmlhboxmathB{\adjustnormalsize\setbox\sizebox=\hbox to\hsize\bgroup\hfill }%
\newcommand\lthtmlvboxmathA{\adjustnormalsize\setbox\sizebox=\vbox\bgroup %
 \let\ifinner=\iffalse \let\)\liih@math }%
\newcommand\lthtmlboxmathZ{\@next\next\@currlist{}{\def\next{\voidb@x}}%
 \expandafter\box\next\egroup}%
\newcommand\lthtmlmathtype[1]{\gdef\lthtmlmathenv{#1}}%
\newcommand\lthtmllogmath{\lthtmltypeout{l2hSize %
:\lthtmlmathenv:\the\ht\sizebox::\the\dp\sizebox::\the\wd\sizebox.\preveqno}}%
\newcommand\lthtmlfigureA[1]{\let\@savefreelist\@freelist
       \lthtmlmathtype{#1}\lthtmlvboxmathA}%
\newcommand\lthtmlpictureA{\bgroup\catcode`\_=8 \lthtmlpictureB}%
\newcommand\lthtmlpictureB[1]{\lthtmlmathtype{#1}\egroup
       \let\@savefreelist\@freelist \lthtmlhboxmathB}%
\newcommand\lthtmlpictureZ[1]{\hfill\lthtmlfigureZ}%
\newcommand\lthtmlfigureZ{\lthtmlboxmathZ\lthtmllogmath\copy\sizebox
       \global\let\@freelist\@savefreelist}%
\newcommand\lthtmldisplayA{\bgroup\catcode`\_=8 \lthtmldisplayAi}%
\newcommand\lthtmldisplayAi[1]{\lthtmlmathtype{#1}\egroup\lthtmlvboxmathA}%
\newcommand\lthtmldisplayB[1]{\edef\preveqno{(\theequation)}%
  \lthtmldisplayA{#1}\let\@eqnnum\relax}%
\newcommand\lthtmldisplayZ{\lthtmlboxmathZ\lthtmllogmath\lthtmlsetmath}%
\newcommand\lthtmlinlinemathA{\bgroup\catcode`\_=8 \lthtmlinlinemathB}
\newcommand\lthtmlinlinemathB[1]{\lthtmlmathtype{#1}\egroup\lthtmlhboxmathA
  \vrule height1.5ex width0pt }%
\newcommand\lthtmlinlineA{\bgroup\catcode`\_=8 \lthtmlinlineB}%
\newcommand\lthtmlinlineB[1]{\lthtmlmathtype{#1}\egroup\lthtmlhboxmathA}%
\newcommand\lthtmlinlineZ{\egroup\expandafter\ifdim\dp\sizebox>0pt %
  \expandafter\centerinlinemath\fi\lthtmllogmath\lthtmlsetinline}
\newcommand\lthtmlinlinemathZ{\egroup\expandafter\ifdim\dp\sizebox>0pt %
  \expandafter\centerinlinemath\fi\lthtmllogmath\lthtmlsetmath}
\newcommand\lthtmlindisplaymathZ{\egroup %
  \centerinlinemath\lthtmllogmath\lthtmlsetmath}
\def\lthtmlsetinline{\hbox{\vrule width.1em \vtop{\vbox{%
  \kern.1em\copy\sizebox}\ifdim\dp\sizebox>0pt\kern.1em\else\kern.3pt\fi
  \ifdim\hsize>\wd\sizebox \hrule depth1pt\fi}}}
\def\lthtmlsetmath{\hbox{\vrule width.1em\kern-.05em\vtop{\vbox{%
  \kern.1em\kern0.8 pt\hbox{\hglue.17em\copy\sizebox\hglue0.8 pt}}\kern.3pt%
  \ifdim\dp\sizebox>0pt\kern.1em\fi \kern0.8 pt%
  \ifdim\hsize>\wd\sizebox \hrule depth1pt\fi}}}
\def\centerinlinemath{%
  \dimen1=\ifdim\ht\sizebox<\dp\sizebox \dp\sizebox\else\ht\sizebox\fi
  \advance\dimen1by.5pt \vrule width0pt height\dimen1 depth\dimen1 
 \dp\sizebox=\dimen1\ht\sizebox=\dimen1\relax}

\def\lthtmlcheckvsize{\ifdim\ht\sizebox<\vsize 
  \ifdim\wd\sizebox<\hsize\expandafter\hfill\fi \expandafter\vfill
  \else\expandafter\vss\fi}%
\providecommand{\selectlanguage}[1]{}%
\makeatletter \tracingstats = 1 


\begin{document}
\pagestyle{empty}\thispagestyle{empty}\lthtmltypeout{}%
\lthtmltypeout{latex2htmlLength hsize=\the\hsize}\lthtmltypeout{}%
\lthtmltypeout{latex2htmlLength vsize=\the\vsize}\lthtmltypeout{}%
\lthtmltypeout{latex2htmlLength hoffset=\the\hoffset}\lthtmltypeout{}%
\lthtmltypeout{latex2htmlLength voffset=\the\voffset}\lthtmltypeout{}%
\lthtmltypeout{latex2htmlLength topmargin=\the\topmargin}\lthtmltypeout{}%
\lthtmltypeout{latex2htmlLength topskip=\the\topskip}\lthtmltypeout{}%
\lthtmltypeout{latex2htmlLength headheight=\the\headheight}\lthtmltypeout{}%
\lthtmltypeout{latex2htmlLength headsep=\the\headsep}\lthtmltypeout{}%
\lthtmltypeout{latex2htmlLength parskip=\the\parskip}\lthtmltypeout{}%
\lthtmltypeout{latex2htmlLength oddsidemargin=\the\oddsidemargin}\lthtmltypeout{}%
\makeatletter
\if@twoside\lthtmltypeout{latex2htmlLength evensidemargin=\the\evensidemargin}%
\else\lthtmltypeout{latex2htmlLength evensidemargin=\the\oddsidemargin}\fi%
\lthtmltypeout{}%
\makeatother
\setcounter{page}{1}
\onecolumn

% !!! IMAGES START HERE !!!

{\newpage\clearpage
\lthtmlfigureA{TitlePage223}%
\begin{TitlePage}
\par
\Title{ParFlow Developer's Manual: Draft}
\SubTitle{Software Revision: 1.1.1.1 }
\SubTitle{\today}
\Author{Steven Ashby, Chuck Baldwin, Bill Bosl, Robert Falgout, Steven Smith}
\par
\end{TitlePage}%
\lthtmlfigureZ
\lthtmlcheckvsize\clearpage}

{\newpage\clearpage
\lthtmlfigureA{CopyrightPage229}%
\begin{CopyrightPage}
\par
\noindent
Copyright \copyright{} 1995 The Regents of the University of California.
\par
\vspace{1em}\noindent
Permission is granted to make and distribute verbatim copies of this
manual provided the copyright notice and this permission notice are
preserved on all copies.
\par
\end{CopyrightPage}%
\lthtmlfigureZ
\lthtmlcheckvsize\clearpage}

\stepcounter{chapter}
\stepcounter{chapter}
\stepcounter{section}
{\newpage\clearpage
\lthtmlfigureA{display295}%
\begin{display}\begin{verbatim}

CVSROOT=/home/casc/repository
PARFLOW_SRC=~/parflow/src
PARFLOW_DIR=~/parflow/exe
PARFLOW_HELP=~parflow/docs
export CVSROOT PARFLOW_SRC PARFLOW_DIR PARFLOW_HELP\end{verbatim}
\end{display}%
\lthtmlfigureZ
\lthtmlcheckvsize\clearpage}

{\newpage\clearpage
\lthtmlfigureA{display305}%
\begin{display}\begin{verbatim}

PARFLOW_DIR=~/parflow/exe.`uname`\end{verbatim}
\end{display}%
\lthtmlfigureZ
\lthtmlcheckvsize\clearpage}

{\newpage\clearpage
\lthtmlfigureA{display316}%
\begin{display}\begin{verbatim}

mkdir ~/parflow
cvs checkout -P parflow
(cd ~/parflow/; ./post_checkout)\end{verbatim}
\end{display}%
\lthtmlfigureZ
\lthtmlcheckvsize\clearpage}

\stepcounter{section}
\stepcounter{section}
\stepcounter{section}
\stepcounter{section}
{\newpage\clearpage
\lthtmlfigureA{display367}%
\begin{display}\begin{verbatim}

(cd /home/casc/parflow/;./install.SunOS)\end{verbatim}
\end{display}%
\lthtmlfigureZ
\lthtmlcheckvsize\clearpage}

{\newpage\clearpage
\lthtmlfigureA{display371}%
\begin{display}\begin{verbatim}

(cd /home/casc/parflow/exe.`uname`/test/; tclsh default\_single.pftcl)\end{verbatim}
\end{display}%
\lthtmlfigureZ
\lthtmlcheckvsize\clearpage}

{\newpage\clearpage
\lthtmlfigureA{display377}%
\begin{display}\begin{verbatim}

rlogin nyx\end{verbatim}
\end{display}%
\lthtmlfigureZ
\lthtmlcheckvsize\clearpage}

{\newpage\clearpage
\lthtmlfigureA{display381}%
\begin{display}\begin{verbatim}

(cd /home/casc/parflow/; ./install.IRIX64)\end{verbatim}
\end{display}%
\lthtmlfigureZ
\lthtmlcheckvsize\clearpage}

{\newpage\clearpage
\lthtmlfigureA{display385}%
\begin{display}\begin{verbatim}

(cd /home/casc/parflow/exe.`uname`/test/; tclsh default\_single.pftcl)\end{verbatim}
\end{display}%
\lthtmlfigureZ
\lthtmlcheckvsize\clearpage}

\stepcounter{chapter}
{\newpage\clearpage
\lthtmlfigureA{display479}%
\begin{display}\begin{verbatim}

$PARFLOW_HELP\end{verbatim}
\end{display}%
\lthtmlfigureZ
\lthtmlcheckvsize\clearpage}

{\newpage\clearpage
\lthtmlfigureA{display492}%
\begin{display}\begin{verbatim}

/home/parflow/html/index.html\end{verbatim}
\end{display}%
\lthtmlfigureZ
\lthtmlcheckvsize\clearpage}

{\newpage\clearpage
\lthtmlfigureA{display499}%
\begin{display}\begin{verbatim}

AVS_HELP_PATH=$PARFLOW_DIR/avs/help\end{verbatim}
\end{display}%
\lthtmlfigureZ
\lthtmlcheckvsize\clearpage}

\stepcounter{section}
{\newpage\clearpage
\lthtmlfigureA{display522}%
\begin{display}\begin{verbatim}

build\end{verbatim}
\end{display}%
\lthtmlfigureZ
\lthtmlcheckvsize\clearpage}

{\newpage\clearpage
\lthtmlfigureA{display527}%
\begin{display}\begin{verbatim}

xdvi manual
ghostview manual.ps\end{verbatim}
\end{display}%
\lthtmlfigureZ
\lthtmlcheckvsize\clearpage}

{\newpage\clearpage
\lthtmlfigureA{display531}%
\begin{display}\begin{verbatim}

netscape manual/manual.html\end{verbatim}
\end{display}%
\lthtmlfigureZ
\lthtmlcheckvsize\clearpage}

{\newpage\clearpage
\lthtmlfigureA{display536}%
\begin{display}\begin{verbatim}

build install\end{verbatim}
\end{display}%
\lthtmlfigureZ
\lthtmlcheckvsize\clearpage}

\stepcounter{subsection}
{\newpage\clearpage
\lthtmlfigureA{display548}%
\begin{display}\begin{verbatim}

\code{<sample-program-code>}
\file{<file-name>}
\kbd{<keyboard-input>}
\key{<keyboard-key-name>}
\samp{<literal-text>}\end{verbatim}
\end{display}%
\lthtmlfigureZ
\lthtmlcheckvsize\clearpage}

{\newpage\clearpage
\lthtmlfigureA{display552}%
\begin{display}\begin{verbatim}

# & ~ _ ^\end{verbatim}
\end{display}%
\lthtmlfigureZ
\lthtmlcheckvsize\clearpage}

{\newpage\clearpage
\lthtmlfigureA{display557}%
\begin{display}\begin{verbatim}

$ % \ { }
\end{verbatim}
\end{display}%
\lthtmlfigureZ
\lthtmlcheckvsize\clearpage}

{\newpage\clearpage
\lthtmlfigureA{display561}%
\begin{display}\begin{verbatim}

\dfn{<term-being-defined>}
\var{<metasyntactic-variable>}\end{verbatim}
\end{display}%
\lthtmlfigureZ
\lthtmlcheckvsize\clearpage}

{\newpage\clearpage
\lthtmlfigureA{display565}%
\begin{display}\begin{verbatim}

\defmac{<macro-name>}(<arguments>)
\deftp{<category>}{<name-of-type>}{<attributes>}
\deftypefn{<classification>}{<data-type>}{<name>}(<arguments>)
\deftypevr{<classification>}{<data-type>}{<name>}\end{verbatim}
\end{display}%
\lthtmlfigureZ
\lthtmlcheckvsize\clearpage}

{\newpage\clearpage
\lthtmlfigureA{display570}%
\begin{display}\begin{verbatim}

\DESCRIPTION
\EXAMPLE
\SEEALSO
\NOTES\end{verbatim}
\end{display}%
\lthtmlfigureZ
\lthtmlcheckvsize\clearpage}

{\newpage\clearpage
\lthtmlfigureA{display576}%
\begin{display}\begin{verbatim}

\begin{deftypefn}{Function}{Region *}{NewRegion}(int \var{size})

%=========================== DESCRIPTION
\DESCRIPTION
Creates and returns a pointer to a new \code{Region} structure
containing \var{size} pointers to new \code{SubregionArray} structures.

%=========================== SEE ALSO
\SEEALSO
\vref{Grid Structures}{Grid Structures}\\
\vref{NewSubregionArray}{NewSubregionArray}\\
\vref{AppendSubregion}{AppendSubregion}\\
\vref{AppendSubregionArray}{AppendSubregionArray}\\
\vref{Gridding}{Gridding}

\end{deftypefn}\end{verbatim}
\end{display}%
\lthtmlfigureZ
\lthtmlcheckvsize\clearpage}

{\newpage\clearpage
\lthtmlfigureA{display580}%
\begin{display}\begin{verbatim}

\vref{<label>}{<title>}\end{verbatim}
\end{display}%
\lthtmlfigureZ
\lthtmlcheckvsize\clearpage}

{\newpage\clearpage
\lthtmlfigureA{display587}%
\begin{display}\begin{verbatim}

\Title{<title}
\SubTitle{<subtitle>}
\Author{<authors>}\end{verbatim}
\end{display}%
\lthtmlfigureZ
\lthtmlcheckvsize\clearpage}

{\newpage\clearpage
\lthtmlfigureA{display594}%
\begin{display}\begin{verbatim}

\cindex{<index string>}
\findex{<index string>}
\kindex{<index string>}
\pindex{<index string>}
\tindex{<index string>}
\vindex{<index string>}\end{verbatim}
\end{display}%
\lthtmlfigureZ
\lthtmlcheckvsize\clearpage}

{\newpage\clearpage
\lthtmlfigureA{display601}%
\begin{display}\begin{verbatim}

\index{\findex{<index string>}}\end{verbatim}
\end{display}%
\lthtmlfigureZ
\lthtmlcheckvsize\clearpage}

{\newpage\clearpage
\lthtmlfigureA{display608}%
\begin{display}\begin{verbatim}

\parflow{}
\xparflow{}
\pftools{}\end{verbatim}
\end{display}%
\lthtmlfigureZ
\lthtmlcheckvsize\clearpage}

{\newpage\clearpage
\lthtmlfigureA{display612}%
\begin{display}\begin{verbatim}

\pfaddanchor{<unique tag>}\end{verbatim}
\end{display}%
\lthtmlfigureZ
\lthtmlcheckvsize\clearpage}

\stepcounter{subsection}
{\newpage\clearpage
\lthtmlfigureA{display621}%
\begin{display}\begin{verbatim}

\pfaddanchor{<unique tag>}\end{verbatim}
\end{display}%
\lthtmlfigureZ
\lthtmlcheckvsize\clearpage}

\stepcounter{section}
{\newpage\clearpage
\lthtmlfigureA{display641}%
\begin{display}\begin{verbatim}

build install\end{verbatim}
\end{display}%
\lthtmlfigureZ
\lthtmlcheckvsize\clearpage}

\stepcounter{section}
\stepcounter{section}
{\newpage\clearpage
\lthtmlfigureA{display664}%
\begin{display}\begin{verbatim}

build install\end{verbatim}
\end{display}%
\lthtmlfigureZ
\lthtmlcheckvsize\clearpage}

\stepcounter{section}
\stepcounter{chapter}
\stepcounter{section}
{\newpage\clearpage
\lthtmlfigureA{display794}%
\begin{display}\begin{verbatim}

PFModule  *PFModuleNewModule(name, (... args ...))
void       PFModuleFreeModule(module)
PFModule  *PFModuleNewInstance(module, (... args ...))
PFModule  *PFModuleReNewInstance(module_instance, (... args ...))
void       PFModuleFreeInstance(module_instance)
int        PFModuleSizeOfTempData(module_instance)
type       PFModuleInvoke(type, module_instance, (... args ...))\end{verbatim}
\end{display}%
\lthtmlfigureZ
\lthtmlcheckvsize\clearpage}

\stepcounter{section}
{\newpage\clearpage
\lthtmlfigureA{display826}%
\begin{display}\begin{verbatim}

type       Name(... args ...)
PFModule  *NameInitInstanceXtra(... args ...)
void       NameFreeInstanceXtra()
PFModule  *NameNewPublicXtra(... args ...)
void       NameFreePublicXtra()
int        NameSizeOfTempData()\end{verbatim}
\end{display}%
\lthtmlfigureZ
\lthtmlcheckvsize\clearpage}

{\newpage\clearpage
\lthtmlfigureA{display851}%
\begin{display}\begin{verbatim}

PFModule  *ThisPFModule\end{verbatim}
\end{display}%
\lthtmlfigureZ
\lthtmlcheckvsize\clearpage}

\stepcounter{section}
\stepcounter{section}
\stepcounter{chapter}
{\newpage\clearpage
\lthtmlfigureA{display944}%
\begin{display}\begin{verbatim}

   char *geom_name;

   geom_name = GetString("Domain.GeomName");

   geom_name = GetStringDefault("Domain.GeomName", "DefaultGeom");\end{verbatim}
\end{display}%
\lthtmlfigureZ
\lthtmlcheckvsize\clearpage}

{\newpage\clearpage
\lthtmlfigureA{display949}%
\begin{display}\begin{verbatim}

/* name is the name that this module was given by it's parent */
PFModule   *FooNewPublicXtra(char *name)
{
   char key[IDB_MAX_KEY_LEN];   /* key name use to query DB */
        

   sprintf(key, "%s.MaxLevels", name); /* construct the key name from
                                        the modules name and the value
                                        we need */
   public_xtra -> max_levels = GetIntDefault(key, 100);
}\end{verbatim}
\end{display}%
\lthtmlfigureZ
\lthtmlcheckvsize\clearpage}

{\newpage\clearpage
\lthtmlfigureA{display954}%
\begin{display}\begin{verbatim}

   char key[IDB_MAX_KEY_LEN];

   char          *switch_name;
   int            switch_value;

   /* Name arrays are helper structures to parse space seperated strings */
   /* A name array can convert from an index to name or from a name      */
   /* to an index.                                                       */
   NameArray      switch_na;

	
   /* Build the name array with two names:                               */
   /*   0 = RedBlackGSPoint                                              */
   /*   1 = WJacobi                                                      */
   switch_na = NA_NewNameArray("RedBlackGSPoint WJacobi");

   sprintf(key, "%s.Smoother", name);
   switch_name = GetStringDefault(key,"RedBlackGSPoint");

   /* Convert the name the user entered into an int to use in the switch */
   /* -1 is returned if the name is not found in the name array          */
   switch_value  = NA_NameToIndex(switch_na, switch_name);
   switch (switch_value)
   {
      case 0:
      {
	 /* key has the name we want to give to this child */
         public_xtra -> smooth = PFModuleNewModule(RedBlackGSPoint, (key));
         break;
      }
      case 1:
      {
         public_xtra -> smooth = PFModuleNewModule(WJacobi, (key));
         break;
      }
      default:
      {
         InputError("Error: Invalid value <%s> for key <%s>\n", switch_name,
                     key);
      }
   }
   NA_FreeNameArray(switch_na);

}\end{verbatim}
\end{display}%
\lthtmlfigureZ
\lthtmlcheckvsize\clearpage}

\stepcounter{chapter}
\stepcounter{section}
{\newpage\clearpage
\lthtmldisplayA{displaymath983}%
\begin{displaymath} 
\frac{\partial}{\partial t} ( \phi S_i)
  ~+~ \nabla \cdot {\vec V}_i
  ~-~ Q_i~=~ 0 ,
\end{displaymath}%
\lthtmldisplayZ
\lthtmlcheckvsize\clearpage}

{\newpage\clearpage
\lthtmldisplayA{displaymath988}%
\begin{displaymath} 
{\vec V}_i~+~ {\lambda}_i\cdot ( \nabla p_i~-~ \rho _i{\vec g}) ~=~ 0 ,
\end{displaymath}%
\lthtmldisplayZ
\lthtmlcheckvsize\clearpage}

{\newpage\clearpage
\lthtmlinlinemathA{tex2html_wrap_inline1339}%
$i = 0, \ldots , n_p- 1$%
\lthtmlinlinemathZ
\lthtmlcheckvsize\clearpage}

{\newpage\clearpage
\lthtmlinlinemathA{tex2html_wrap_inline1341}%
$(n_p\in \{1,2,3\})$%
\lthtmlinlinemathZ
\lthtmlcheckvsize\clearpage}

{\newpage\clearpage
\lthtmlinlinemathA{tex2html_wrap_indisplay4638}%
$\displaystyle {\lambda}_i$%
\lthtmlindisplaymathZ
\lthtmlcheckvsize\clearpage}

{\newpage\clearpage
\lthtmlinlinemathA{tex2html_wrap_indisplay4639}%
$\textstyle =$%
\lthtmlindisplaymathZ
\lthtmlcheckvsize\clearpage}

{\newpage\clearpage
\lthtmlinlinemathA{tex2html_wrap_indisplay4640}%
$\displaystyle \frac{{\bar k}k_{ri}}{\mu _i} ,$%
\lthtmlindisplaymathZ
\lthtmlcheckvsize\clearpage}

{\newpage\clearpage
\lthtmlinlinemathA{tex2html_wrap_indisplay4641}%
$\displaystyle {\vec g}$%
\lthtmlindisplaymathZ
\lthtmlcheckvsize\clearpage}

{\newpage\clearpage
\lthtmlinlinemathA{tex2html_wrap_indisplay4643}%
$\displaystyle [ 0, 0, -g ]^T ,$%
\lthtmlindisplaymathZ
\lthtmlcheckvsize\clearpage}

{\newpage\clearpage
\lthtmlinlinemathA{tex2html_wrap_inline1343}%
$\phi ({\vec x},t)$%
\lthtmlinlinemathZ
\lthtmlcheckvsize\clearpage}

{\newpage\clearpage
\lthtmlinlinemathA{tex2html_wrap_inline1345}%
$S_i({\vec x},t)$%
\lthtmlinlinemathZ
\lthtmlcheckvsize\clearpage}

{\newpage\clearpage
\lthtmlinlinemathA{tex2html_wrap_inline1347}%
${\vec V}_i({\vec x},t)$%
\lthtmlinlinemathZ
\lthtmlcheckvsize\clearpage}

{\newpage\clearpage
\lthtmlinlinemathA{tex2html_wrap_inline1349}%
$L T^{-1}$%
\lthtmlinlinemathZ
\lthtmlcheckvsize\clearpage}

{\newpage\clearpage
\lthtmlinlinemathA{tex2html_wrap_inline1351}%
$Q_i({\vec x},t)$%
\lthtmlinlinemathZ
\lthtmlcheckvsize\clearpage}

{\newpage\clearpage
\lthtmlinlinemathA{tex2html_wrap_inline1353}%
$T^{-1}$%
\lthtmlinlinemathZ
\lthtmlcheckvsize\clearpage}

{\newpage\clearpage
\lthtmlinlinemathA{tex2html_wrap_inline1355}%
${\lambda}_i$%
\lthtmlinlinemathZ
\lthtmlcheckvsize\clearpage}

{\newpage\clearpage
\lthtmlinlinemathA{tex2html_wrap_inline1357}%
$L^{3} T M^{-1}$%
\lthtmlinlinemathZ
\lthtmlcheckvsize\clearpage}

{\newpage\clearpage
\lthtmlinlinemathA{tex2html_wrap_inline1359}%
$p_i({\vec x},t)$%
\lthtmlinlinemathZ
\lthtmlcheckvsize\clearpage}

{\newpage\clearpage
\lthtmlinlinemathA{tex2html_wrap_inline1361}%
$M L^{-1} T^{-2}$%
\lthtmlinlinemathZ
\lthtmlcheckvsize\clearpage}

{\newpage\clearpage
\lthtmlinlinemathA{tex2html_wrap_inline1363}%
$\rho _i$%
\lthtmlinlinemathZ
\lthtmlcheckvsize\clearpage}

{\newpage\clearpage
\lthtmlinlinemathA{tex2html_wrap_inline1365}%
$M L^{-3}$%
\lthtmlinlinemathZ
\lthtmlcheckvsize\clearpage}

{\newpage\clearpage
\lthtmlinlinemathA{tex2html_wrap_inline1367}%
${\vec g}$%
\lthtmlinlinemathZ
\lthtmlcheckvsize\clearpage}

{\newpage\clearpage
\lthtmlinlinemathA{tex2html_wrap_inline1369}%
$L T^{-2}$%
\lthtmlinlinemathZ
\lthtmlcheckvsize\clearpage}

{\newpage\clearpage
\lthtmlinlinemathA{tex2html_wrap_inline1371}%
${\bar k}({\vec x},t)$%
\lthtmlinlinemathZ
\lthtmlcheckvsize\clearpage}

{\newpage\clearpage
\lthtmlinlinemathA{tex2html_wrap_inline1373}%
$L^{2}$%
\lthtmlinlinemathZ
\lthtmlcheckvsize\clearpage}

{\newpage\clearpage
\lthtmlinlinemathA{tex2html_wrap_inline1375}%
$k_{ri}({\vec x},t)$%
\lthtmlinlinemathZ
\lthtmlcheckvsize\clearpage}

{\newpage\clearpage
\lthtmlinlinemathA{tex2html_wrap_inline1377}%
$\mu _i$%
\lthtmlinlinemathZ
\lthtmlcheckvsize\clearpage}

{\newpage\clearpage
\lthtmlinlinemathA{tex2html_wrap_inline1379}%
$M L^{-1} T^{-1}$%
\lthtmlinlinemathZ
\lthtmlcheckvsize\clearpage}

{\newpage\clearpage
\lthtmlinlinemathA{tex2html_wrap_inline1381}%
$g$%
\lthtmlinlinemathZ
\lthtmlcheckvsize\clearpage}

{\newpage\clearpage
\lthtmlinlinemathA{tex2html_wrap_inline1385}%
$\phi $%
\lthtmlinlinemathZ
\lthtmlcheckvsize\clearpage}

{\newpage\clearpage
\lthtmlinlinemathA{tex2html_wrap_inline1387}%
$S_i$%
\lthtmlinlinemathZ
\lthtmlcheckvsize\clearpage}

{\newpage\clearpage
\lthtmlinlinemathA{tex2html_wrap_inline1389}%
$i$%
\lthtmlinlinemathZ
\lthtmlcheckvsize\clearpage}

{\newpage\clearpage
\lthtmlinlinemathA{tex2html_wrap_inline1391}%
$0 \le \phi \le 1$%
\lthtmlinlinemathZ
\lthtmlcheckvsize\clearpage}

{\newpage\clearpage
\lthtmlinlinemathA{tex2html_wrap_inline1393}%
$0 \le S_i\le 1$%
\lthtmlinlinemathZ
\lthtmlcheckvsize\clearpage}

{\newpage\clearpage
\lthtmlinlinemathA{tex2html_wrap_inline1395}%
${\bar k}$%
\lthtmlinlinemathZ
\lthtmlcheckvsize\clearpage}

{\newpage\clearpage
\lthtmlinlinemathA{tex2html_wrap_inline1397}%
$3 \times 3$%
\lthtmlinlinemathZ
\lthtmlcheckvsize\clearpage}

{\newpage\clearpage
\lthtmlinlinemathA{tex2html_wrap_inline1403}%
$k_{ri}(S_i)$%
\lthtmlinlinemathZ
\lthtmlcheckvsize\clearpage}

{\newpage\clearpage
\lthtmlinlinemathA{tex2html_wrap_inline1405}%
${\vec v}_i$%
\lthtmlinlinemathZ
\lthtmlcheckvsize\clearpage}

{\newpage\clearpage
\lthtmldisplayA{displaymath1018}%
\begin{displaymath} 
{\vec V}_i= \phi S_i{\vec v}_i.
\end{displaymath}%
\lthtmldisplayZ
\lthtmlcheckvsize\clearpage}

{\newpage\clearpage
\lthtmlinlinemathA{tex2html_wrap_inline1407}%
$n_p$%
\lthtmlinlinemathZ
\lthtmlcheckvsize\clearpage}

{\newpage\clearpage
\lthtmldisplayA{displaymath1022}%
\begin{displaymath} 
\sum_i S_i= 1 ,
\end{displaymath}%
\lthtmldisplayZ
\lthtmlcheckvsize\clearpage}

{\newpage\clearpage
\lthtmldisplayA{displaymath1025}%
\begin{displaymath} 
p_{i0} ~=~ p_{i0} ( S_0 ) ,
~~~~~~ i = 1 , \ldots , n_p- 1 .
\end{displaymath}%
\lthtmldisplayZ
\lthtmlcheckvsize\clearpage}

{\newpage\clearpage
\lthtmlinlinemathA{tex2html_wrap_inline1409}%
$p_{ij} = p_i - p_j$%
\lthtmlinlinemathZ
\lthtmlcheckvsize\clearpage}

{\newpage\clearpage
\lthtmlinlinemathA{tex2html_wrap_inline1413}%
$j$%
\lthtmlinlinemathZ
\lthtmlcheckvsize\clearpage}

{\newpage\clearpage
\lthtmlinlinemathA{tex2html_wrap_inline1415}%
$3 n_p$%
\lthtmlinlinemathZ
\lthtmlcheckvsize\clearpage}

{\newpage\clearpage
\lthtmlinlinemathA{tex2html_wrap_inline1419}%
$S_i, {\vec V}_i$%
\lthtmlinlinemathZ
\lthtmlcheckvsize\clearpage}

{\newpage\clearpage
\lthtmlinlinemathA{tex2html_wrap_inline1421}%
$p_i$%
\lthtmlinlinemathZ
\lthtmlcheckvsize\clearpage}

{\newpage\clearpage
\lthtmlinlinemathA{tex2html_wrap_inline1423}%
${\lambda}_T$%
\lthtmlinlinemathZ
\lthtmlcheckvsize\clearpage}

{\newpage\clearpage
\lthtmlinlinemathA{tex2html_wrap_inline1425}%
${\vec V}_T$%
\lthtmlinlinemathZ
\lthtmlcheckvsize\clearpage}

{\newpage\clearpage
\lthtmlinlinemathA{tex2html_wrap_indisplay4732}%
$\displaystyle {\lambda}_T~=~ \sum_{i} {\lambda}_i,$%
\lthtmlindisplaymathZ
\lthtmlcheckvsize\clearpage}

{\newpage\clearpage
\lthtmlinlinemathA{tex2html_wrap_indisplay4733}%
$\displaystyle {\vec V}_T~=~ \sum_{i} {\vec V}_i.$%
\lthtmlindisplaymathZ
\lthtmlcheckvsize\clearpage}

{\newpage\clearpage
\lthtmlinlinemathA{tex2html_wrap_inline1427}%
$p_0$%
\lthtmlinlinemathZ
\lthtmlcheckvsize\clearpage}

{\newpage\clearpage
\lthtmldisplayA{displaymath1044}%
\begin{displaymath} 
-~ \sum_{i}
  \left \{
    \nabla \cdot {\lambda}_i
      \left ( \nabla ( p_0 ~+~ p_{i0} ) ~-~ \rho _i{\vec g}\right )
    ~+~
    Q_i
  \right \}
~=~ 0 .
\end{displaymath}%
\lthtmldisplayZ
\lthtmlcheckvsize\clearpage}

{\newpage\clearpage
\lthtmlinlinemathA{tex2html_wrap_inline1429}%
$n_p- 1$%
\lthtmlinlinemathZ
\lthtmlcheckvsize\clearpage}

{\newpage\clearpage
\lthtmldisplayA{displaymath1049}%
\begin{displaymath} 
\frac{\partial}{\partial t} ( \phi S_i)
~+~
\nabla \cdot
  \left (
     \frac{{\lambda}_i}{{\lambda}_T} {\vec V}_T~+~
     \sum_{j \neq i} \frac{{\lambda}_i{\lambda}_j}{{\lambda}_T} ( \rho _i - \rho _j ) {\vec g}
  \right )
~+~
\sum_{j \neq i} \nabla \cdot
    \frac{{\lambda}_i{\lambda}_j}{{\lambda}_T} \nabla p_{ji}
~-~ Q_i
~=~ 0 .
\end{displaymath}%
\lthtmldisplayZ
\lthtmlcheckvsize\clearpage}

{\newpage\clearpage
\lthtmlinlinemathA{tex2html_wrap_inline1433}%
$p_{ji}$%
\lthtmlinlinemathZ
\lthtmlcheckvsize\clearpage}

{\newpage\clearpage
\lthtmldisplayA{displaymath1066}%
\begin{displaymath} 
p_{ji} ~=~ p_{j0} ~-~ p_{i0} ,
\end{displaymath}%
\lthtmldisplayZ
\lthtmlcheckvsize\clearpage}

{\newpage\clearpage
\lthtmlinlinemathA{tex2html_wrap_inline1435}%
$p_{ii} = 0$%
\lthtmlinlinemathZ
\lthtmlcheckvsize\clearpage}

\stepcounter{section}
{\newpage\clearpage
\lthtmlinlinemathA{tex2html_wrap_indisplay4748}%
$\displaystyle \left ( \frac{\partial}{\partial t} (\phi c_{i,j}) ~+~ \lambda_j ~ \phi c_{i,j}\right )$%
\lthtmlindisplaymathZ
\lthtmlcheckvsize\clearpage}

{\newpage\clearpage
\lthtmlinlinemathA{tex2html_wrap_indisplay4749}%
$\textstyle +$%
\lthtmlindisplaymathZ
\lthtmlcheckvsize\clearpage}

{\newpage\clearpage
\lthtmlinlinemathA{tex2html_wrap_indisplay4750}%
$\displaystyle \nabla \cdot \left ( c_{i,j}{\vec V}_i\right )$%
\lthtmlindisplaymathZ
\lthtmlcheckvsize\clearpage}

{\newpage\clearpage
\lthtmlinlinemathA{tex2html_wrap_indisplay4752}%
$\displaystyle -\left ( \frac{\partial}{\partial t} ((1 - \phi ) \rho_{s}F_{i,j}) ~+~  \lambda_j ~ (1 - \phi ) \rho_{s}F_{i,j}\right )$%
\lthtmlindisplaymathZ
\lthtmlcheckvsize\clearpage}

{\newpage\clearpage
\lthtmlinlinemathA{tex2html_wrap_indisplay4754}%
$\displaystyle \sum_{k}^{n_{I}} \gamma^{I;i}_{k}\chi_{\Omega^{I}_{k}} \left ( c_{i,j}- {\bar c}^{k}_{ij}\right ) ~-~ \sum_{k}^{n_{E}} \gamma^{E;i}_{k}\chi_{\Omega^{E}_{k}} c_{i,j}$%
\lthtmlindisplaymathZ
\lthtmlcheckvsize\clearpage}

{\newpage\clearpage
\lthtmlinlinemathA{tex2html_wrap_inline1449}%
$j = 0, \ldots , n_c- 1$%
\lthtmlinlinemathZ
\lthtmlcheckvsize\clearpage}

{\newpage\clearpage
\lthtmlinlinemathA{tex2html_wrap_inline1451}%
$c_{i,j}$%
\lthtmlinlinemathZ
\lthtmlcheckvsize\clearpage}

{\newpage\clearpage
\lthtmlinlinemathA{tex2html_wrap_inline1457}%
$\chi_A$%
\lthtmlinlinemathZ
\lthtmlcheckvsize\clearpage}

{\newpage\clearpage
\lthtmlinlinemathA{tex2html_wrap_inline1459}%
$A$%
\lthtmlinlinemathZ
\lthtmlcheckvsize\clearpage}

{\newpage\clearpage
\lthtmlinlinemathA{tex2html_wrap_inline1461}%
$\chi_A(x) = 1$%
\lthtmlinlinemathZ
\lthtmlcheckvsize\clearpage}

{\newpage\clearpage
\lthtmlinlinemathA{tex2html_wrap_inline1463}%
$x \in A$%
\lthtmlinlinemathZ
\lthtmlcheckvsize\clearpage}

{\newpage\clearpage
\lthtmlinlinemathA{tex2html_wrap_inline1465}%
$\chi_A(x) = 0$%
\lthtmlinlinemathZ
\lthtmlcheckvsize\clearpage}

{\newpage\clearpage
\lthtmlinlinemathA{tex2html_wrap_inline1467}%
$x \not\in A$%
\lthtmlinlinemathZ
\lthtmlcheckvsize\clearpage}

{\newpage\clearpage
\lthtmlinlinemathA{tex2html_wrap_inline1469}%
$\phi ({\vec x})$%
\lthtmlinlinemathZ
\lthtmlcheckvsize\clearpage}

{\newpage\clearpage
\lthtmlinlinemathA{tex2html_wrap_inline1471}%
$c_{i,j}({\vec x},t)$%
\lthtmlinlinemathZ
\lthtmlcheckvsize\clearpage}

{\newpage\clearpage
\lthtmlinlinemathA{tex2html_wrap_inline1477}%
$\lambda_j $%
\lthtmlinlinemathZ
\lthtmlcheckvsize\clearpage}

{\newpage\clearpage
\lthtmlinlinemathA{tex2html_wrap_inline1481}%
$\rho_{s}({\vec x})$%
\lthtmlinlinemathZ
\lthtmlcheckvsize\clearpage}

{\newpage\clearpage
\lthtmlinlinemathA{tex2html_wrap_inline1485}%
$F_{i,j}({\vec x}, t)$%
\lthtmlinlinemathZ
\lthtmlcheckvsize\clearpage}

{\newpage\clearpage
\lthtmlinlinemathA{tex2html_wrap_inline1487}%
$L^{3} M^{-1}$%
\lthtmlinlinemathZ
\lthtmlcheckvsize\clearpage}

{\newpage\clearpage
\lthtmlinlinemathA{tex2html_wrap_inline1489}%
$n_{I}$%
\lthtmlinlinemathZ
\lthtmlcheckvsize\clearpage}

{\newpage\clearpage
\lthtmlinlinemathA{tex2html_wrap_inline1491}%
$\gamma^{I;i}_{k}(t)$%
\lthtmlinlinemathZ
\lthtmlcheckvsize\clearpage}

{\newpage\clearpage
\lthtmlinlinemathA{tex2html_wrap_inline1495}%
$\Omega^{I}_{k}({\vec x})$%
\lthtmlinlinemathZ
\lthtmlcheckvsize\clearpage}

{\newpage\clearpage
\lthtmlinlinemathA{tex2html_wrap_inline1497}%
${\bar c}^{k}_{ij}()$%
\lthtmlinlinemathZ
\lthtmlcheckvsize\clearpage}

{\newpage\clearpage
\lthtmlinlinemathA{tex2html_wrap_inline1499}%
$n_{E}$%
\lthtmlinlinemathZ
\lthtmlcheckvsize\clearpage}

{\newpage\clearpage
\lthtmlinlinemathA{tex2html_wrap_inline1501}%
$\gamma^{E;i}_{k}(t)$%
\lthtmlinlinemathZ
\lthtmlcheckvsize\clearpage}

{\newpage\clearpage
\lthtmlinlinemathA{tex2html_wrap_inline1505}%
$\Omega^{E}_{k}({\vec x})$%
\lthtmlinlinemathZ
\lthtmlcheckvsize\clearpage}

{\newpage\clearpage
\lthtmlinlinemathA{tex2html_wrap_inline1507}%
$F_{i,j}$%
\lthtmlinlinemathZ
\lthtmlcheckvsize\clearpage}

{\newpage\clearpage
\lthtmldisplayA{displaymath1111}%
\begin{displaymath} 
F_{i,j}= K_{d;j}c_{i,j},
\end{displaymath}%
\lthtmldisplayZ
\lthtmlcheckvsize\clearpage}

{\newpage\clearpage
\lthtmlinlinemathA{tex2html_wrap_inline1509}%
$K_{d;j}$%
\lthtmlinlinemathZ
\lthtmlcheckvsize\clearpage}

{\newpage\clearpage
\lthtmlinlinemathA{tex2html_wrap_indisplay4836}%
$\displaystyle (\phi + (1 - \phi ) \rho_{s}K_{d;j}) \frac{\partial}{\partial t} c_{i,j}$%
\lthtmlindisplaymathZ
\lthtmlcheckvsize\clearpage}

{\newpage\clearpage
\lthtmlinlinemathA{tex2html_wrap_indisplay4840}%
$\displaystyle -~(\phi + (1 - \phi ) \rho_{s}K_{d;j}) \lambda_j c_{i,j}$%
\lthtmlindisplaymathZ
\lthtmlcheckvsize\clearpage}

\stepcounter{section}
{\newpage\clearpage
\lthtmldisplayA{displaymath1134}%
\begin{displaymath} 
{\vec V}_i~+~ {\bar K}_i\cdot ( \nabla h_i~-~ \frac{\rho _i}{\gamma } {\vec g}) ~=~ 0 ,
\end{displaymath}%
\lthtmldisplayZ
\lthtmlcheckvsize\clearpage}

{\newpage\clearpage
\lthtmlinlinemathA{tex2html_wrap_indisplay4846}%
$\displaystyle {\bar K}_i$%
\lthtmlindisplaymathZ
\lthtmlcheckvsize\clearpage}

{\newpage\clearpage
\lthtmlinlinemathA{tex2html_wrap_indisplay4848}%
$\displaystyle \gamma {\lambda}_i,$%
\lthtmlindisplaymathZ
\lthtmlcheckvsize\clearpage}

{\newpage\clearpage
\lthtmlinlinemathA{tex2html_wrap_indisplay4849}%
$\displaystyle h_i$%
\lthtmlindisplaymathZ
\lthtmlcheckvsize\clearpage}

{\newpage\clearpage
\lthtmlinlinemathA{tex2html_wrap_indisplay4851}%
$\displaystyle ( p_i~-~ \bar{p}) / \gamma .$%
\lthtmlindisplaymathZ
\lthtmlcheckvsize\clearpage}

{\newpage\clearpage
\lthtmlinlinemathA{tex2html_wrap_inline1513}%
${\vec V}_i$%
\lthtmlinlinemathZ
\lthtmlcheckvsize\clearpage}

{\newpage\clearpage
\lthtmlinlinemathA{tex2html_wrap_inline1517}%
${\bar K}_i$%
\lthtmlinlinemathZ
\lthtmlcheckvsize\clearpage}

{\newpage\clearpage
\lthtmlinlinemathA{tex2html_wrap_inline1521}%
$h_i$%
\lthtmlinlinemathZ
\lthtmlcheckvsize\clearpage}

{\newpage\clearpage
\lthtmlinlinemathA{tex2html_wrap_inline1523}%
$L$%
\lthtmlinlinemathZ
\lthtmlcheckvsize\clearpage}

{\newpage\clearpage
\lthtmlinlinemathA{tex2html_wrap_inline1525}%
$\gamma $%
\lthtmlinlinemathZ
\lthtmlcheckvsize\clearpage}

{\newpage\clearpage
\lthtmlinlinemathA{tex2html_wrap_inline1527}%
$M L^{-2} T^{-2}$%
\lthtmlinlinemathZ
\lthtmlcheckvsize\clearpage}

{\newpage\clearpage
\lthtmldisplayA{displaymath1157}%
\begin{displaymath} 
-~ \sum_{i}
  \left \{
    \nabla \cdot {\bar K}_i
      \left ( \nabla ( h_0 ~+~ h_{i0} ) ~-~
        \frac{\rho _i}{\gamma } {\vec g}\right )
    ~+~
    Q_i
  \right \}
~=~ 0 ,
\end{displaymath}%
\lthtmldisplayZ
\lthtmlcheckvsize\clearpage}

{\newpage\clearpage
\lthtmldisplayA{displaymath1164}%
\begin{displaymath} 
\frac{\partial}{\partial t} ( \phi S_i)
~+~
\nabla \cdot
  \left (
     \frac{{\bar K}_i}{{\bar K}_T} {\vec V}_T~+~
     \sum_{j \neq i} \frac{{\bar K}_i{\bar K}_j}{{\bar K}_T}
       \left ( \frac{\rho _i}{\gamma } - \frac{\rho _j}{\gamma } \right ) {\vec g}
  \right )
~+~
\sum_{j \neq i} \nabla \cdot
    \frac{{\bar K}_i{\bar K}_j}{{\bar K}_T} \nabla h_{ji}
~-~ Q_i
~=~ 0 .
\end{displaymath}%
\lthtmldisplayZ
\lthtmlcheckvsize\clearpage}

{\newpage\clearpage
\lthtmlinlinemathA{tex2html_wrap_inline1535}%
$(I)$%
\lthtmlinlinemathZ
\lthtmlcheckvsize\clearpage}

{\newpage\clearpage
\lthtmlinlinemathA{tex2html_wrap_inline1537}%
$\gamma = \rho _0 g$%
\lthtmlinlinemathZ
\lthtmlcheckvsize\clearpage}

{\newpage\clearpage
\lthtmlinlinemathA{tex2html_wrap_indisplay4889}%
$\displaystyle {\bar k}(I)$%
\lthtmlindisplaymathZ
\lthtmlcheckvsize\clearpage}

{\newpage\clearpage
\lthtmlinlinemathA{tex2html_wrap_indisplay4891}%
$\displaystyle \gamma {\bar k}/ \mu _0 ;$%
\lthtmlindisplaymathZ
\lthtmlcheckvsize\clearpage}

{\newpage\clearpage
\lthtmlinlinemathA{tex2html_wrap_indisplay4892}%
$\displaystyle \mu _i(I)$%
\lthtmlindisplaymathZ
\lthtmlcheckvsize\clearpage}

{\newpage\clearpage
\lthtmlinlinemathA{tex2html_wrap_indisplay4894}%
$\displaystyle \mu _i/ \mu _0 ;$%
\lthtmlindisplaymathZ
\lthtmlcheckvsize\clearpage}

{\newpage\clearpage
\lthtmlinlinemathA{tex2html_wrap_indisplay4895}%
$\displaystyle p_i(I)$%
\lthtmlindisplaymathZ
\lthtmlcheckvsize\clearpage}

{\newpage\clearpage
\lthtmlinlinemathA{tex2html_wrap_indisplay4897}%
$\displaystyle h_i;$%
\lthtmlindisplaymathZ
\lthtmlcheckvsize\clearpage}

{\newpage\clearpage
\lthtmlinlinemathA{tex2html_wrap_indisplay4898}%
$\displaystyle \rho _i(I)$%
\lthtmlindisplaymathZ
\lthtmlcheckvsize\clearpage}

{\newpage\clearpage
\lthtmlinlinemathA{tex2html_wrap_indisplay4900}%
$\displaystyle \rho _i/ \rho _0 ;$%
\lthtmlindisplaymathZ
\lthtmlcheckvsize\clearpage}

{\newpage\clearpage
\lthtmlinlinemathA{tex2html_wrap_indisplay4901}%
$\displaystyle g (I)$%
\lthtmlindisplaymathZ
\lthtmlcheckvsize\clearpage}

{\newpage\clearpage
\lthtmlinlinemathA{tex2html_wrap_indisplay4903}%
$\displaystyle 1 .$%
\lthtmlindisplaymathZ
\lthtmlcheckvsize\clearpage}

{\newpage\clearpage
\lthtmlinlinemathA{tex2html_wrap_inline1539}%
${\bar k}(I)$%
\lthtmlinlinemathZ
\lthtmlcheckvsize\clearpage}

{\newpage\clearpage
\lthtmlinlinemathA{tex2html_wrap_inline1541}%
${\bar K}_0$%
\lthtmlinlinemathZ
\lthtmlcheckvsize\clearpage}

{\newpage\clearpage
\lthtmlinlinemathA{tex2html_wrap_inline1543}%
$k_{r0} = 1$%
\lthtmlinlinemathZ
\lthtmlcheckvsize\clearpage}

\stepcounter{chapter}
\stepcounter{section}
{\newpage\clearpage
\lthtmlinlinemathA{tex2html_wrap_inline1676}%
$x_0$%
\lthtmlinlinemathZ
\lthtmlcheckvsize\clearpage}

{\newpage\clearpage
\lthtmlinlinemathA{tex2html_wrap_inline1678}%
$y_0$%
\lthtmlinlinemathZ
\lthtmlcheckvsize\clearpage}

{\newpage\clearpage
\lthtmlinlinemathA{tex2html_wrap_inline1680}%
$z_0$%
\lthtmlinlinemathZ
\lthtmlcheckvsize\clearpage}

{\newpage\clearpage
\lthtmlinlinemathA{tex2html_wrap_inline1682}%
$N_x$%
\lthtmlinlinemathZ
\lthtmlcheckvsize\clearpage}

{\newpage\clearpage
\lthtmlinlinemathA{tex2html_wrap_inline1684}%
$N_y$%
\lthtmlinlinemathZ
\lthtmlcheckvsize\clearpage}

{\newpage\clearpage
\lthtmlinlinemathA{tex2html_wrap_inline1686}%
$N_z$%
\lthtmlinlinemathZ
\lthtmlcheckvsize\clearpage}

{\newpage\clearpage
\lthtmlinlinemathA{tex2html_wrap_inline1688}%
$\Delta x$%
\lthtmlinlinemathZ
\lthtmlcheckvsize\clearpage}

{\newpage\clearpage
\lthtmlinlinemathA{tex2html_wrap_inline1690}%
$\Delta y$%
\lthtmlinlinemathZ
\lthtmlcheckvsize\clearpage}

{\newpage\clearpage
\lthtmlinlinemathA{tex2html_wrap_inline1692}%
$\Delta z$%
\lthtmlinlinemathZ
\lthtmlcheckvsize\clearpage}

{\newpage\clearpage
\lthtmlinlinemathA{tex2html_wrap_inline1694}%
$l$%
\lthtmlinlinemathZ
\lthtmlcheckvsize\clearpage}

{\newpage\clearpage
\lthtmlinlinemathA{tex2html_wrap_inline1696}%
$(r_{l,x}, r_{l,y}, r_{l,z})$%
\lthtmlinlinemathZ
\lthtmlcheckvsize\clearpage}

{\newpage\clearpage
\lthtmlinlinemathA{tex2html_wrap_inline1698}%
$(x_{i,l}, y_{j,l}, z_{k,l})$%
\lthtmlinlinemathZ
\lthtmlcheckvsize\clearpage}

{\newpage\clearpage
\lthtmlinlinemathA{tex2html_wrap_inline1700}%
$(i, j, k)$%
\lthtmlinlinemathZ
\lthtmlcheckvsize\clearpage}

{\newpage\clearpage
\lthtmlinlinemathA{tex2html_wrap_indisplay4924}%
$\displaystyle x_{i,l}$%
\lthtmlindisplaymathZ
\lthtmlcheckvsize\clearpage}

{\newpage\clearpage
\lthtmlinlinemathA{tex2html_wrap_indisplay4926}%
$\displaystyle x_0 + i \left( {{\Delta x} \over {2^{r_{l,x}}}} \right)$%
\lthtmlindisplaymathZ
\lthtmlcheckvsize\clearpage}

{\newpage\clearpage
\lthtmlinlinemathA{tex2html_wrap_indisplay4927}%
$\displaystyle y_{j,l}$%
\lthtmlindisplaymathZ
\lthtmlcheckvsize\clearpage}

{\newpage\clearpage
\lthtmlinlinemathA{tex2html_wrap_indisplay4929}%
$\displaystyle y_0 + j \left( {{\Delta y} \over {2^{r_{l,y}}}} \right)$%
\lthtmlindisplaymathZ
\lthtmlcheckvsize\clearpage}

{\newpage\clearpage
\lthtmlinlinemathA{tex2html_wrap_indisplay4930}%
$\displaystyle z_{k,l}$%
\lthtmlindisplaymathZ
\lthtmlcheckvsize\clearpage}

{\newpage\clearpage
\lthtmlinlinemathA{tex2html_wrap_indisplay4932}%
$\displaystyle z_0 + k \left( {{\Delta z} \over {2^{r_{l,z}}}} \right)$%
\lthtmlindisplaymathZ
\lthtmlcheckvsize\clearpage}

{\newpage\clearpage
\lthtmlinlinemathA{tex2html_wrap_inline1704}%
$r_{l,x} + r_{l,y} + r_{l,z}$%
\lthtmlinlinemathZ
\lthtmlcheckvsize\clearpage}

{\newpage\clearpage
\lthtmlinlinemathA{tex2html_wrap_inline1708}%
$m$%
\lthtmlinlinemathZ
\lthtmlcheckvsize\clearpage}

{\newpage\clearpage
\lthtmlinlinemathA{tex2html_wrap_indisplay4937}%
$\displaystyle r_{l,x}$%
\lthtmlindisplaymathZ
\lthtmlcheckvsize\clearpage}

{\newpage\clearpage
\lthtmlinlinemathA{tex2html_wrap_indisplay4938}%
$\textstyle >=$%
\lthtmlindisplaymathZ
\lthtmlcheckvsize\clearpage}

{\newpage\clearpage
\lthtmlinlinemathA{tex2html_wrap_indisplay4939}%
$\displaystyle r_{m,x},$%
\lthtmlindisplaymathZ
\lthtmlcheckvsize\clearpage}

{\newpage\clearpage
\lthtmlinlinemathA{tex2html_wrap_indisplay4940}%
$\displaystyle r_{l,y}$%
\lthtmlindisplaymathZ
\lthtmlcheckvsize\clearpage}

{\newpage\clearpage
\lthtmlinlinemathA{tex2html_wrap_indisplay4942}%
$\displaystyle r_{m,y},$%
\lthtmlindisplaymathZ
\lthtmlcheckvsize\clearpage}

{\newpage\clearpage
\lthtmlinlinemathA{tex2html_wrap_indisplay4943}%
$\displaystyle r_{l,z}$%
\lthtmlindisplaymathZ
\lthtmlcheckvsize\clearpage}

{\newpage\clearpage
\lthtmlinlinemathA{tex2html_wrap_indisplay4945}%
$\displaystyle r_{m,z},$%
\lthtmlindisplaymathZ
\lthtmlcheckvsize\clearpage}

{\newpage\clearpage
\lthtmlinlinemathA{tex2html_wrap_inline1712}%
$i_{l,x}$%
\lthtmlinlinemathZ
\lthtmlcheckvsize\clearpage}

{\newpage\clearpage
\lthtmlinlinemathA{tex2html_wrap_inline1714}%
$i_{l,y}$%
\lthtmlinlinemathZ
\lthtmlcheckvsize\clearpage}

{\newpage\clearpage
\lthtmlinlinemathA{tex2html_wrap_inline1716}%
$i_{l,z}$%
\lthtmlinlinemathZ
\lthtmlcheckvsize\clearpage}

{\newpage\clearpage
\lthtmlinlinemathA{tex2html_wrap_inline1718}%
$n_{l,x}$%
\lthtmlinlinemathZ
\lthtmlcheckvsize\clearpage}

{\newpage\clearpage
\lthtmlinlinemathA{tex2html_wrap_inline1720}%
$n_{l,y}$%
\lthtmlinlinemathZ
\lthtmlcheckvsize\clearpage}

{\newpage\clearpage
\lthtmlinlinemathA{tex2html_wrap_inline1722}%
$n_{l,z}$%
\lthtmlinlinemathZ
\lthtmlcheckvsize\clearpage}

{\newpage\clearpage
\lthtmlinlinemathA{tex2html_wrap_inline1724}%
$s_{l,x}$%
\lthtmlinlinemathZ
\lthtmlcheckvsize\clearpage}

{\newpage\clearpage
\lthtmlinlinemathA{tex2html_wrap_inline1726}%
$s_{l,y}$%
\lthtmlinlinemathZ
\lthtmlcheckvsize\clearpage}

{\newpage\clearpage
\lthtmlinlinemathA{tex2html_wrap_inline1728}%
$s_{l,z}$%
\lthtmlinlinemathZ
\lthtmlcheckvsize\clearpage}

{\newpage\clearpage
\lthtmlinlinemathA{tex2html_wrap_indisplay4957}%
$\displaystyle \{ i_{l,x} + i s_{l,x} : i = 0, 1, \ldots, (n_{l,x}-1) \} \otimes$%
\lthtmlindisplaymathZ
\lthtmlcheckvsize\clearpage}

{\newpage\clearpage
\lthtmlinlinemathA{tex2html_wrap_indisplay4958}%
$\displaystyle \{ i_{l,y} + i s_{l,y} : i = 0, 1, \ldots, (n_{l,y}-1) \} \otimes$%
\lthtmlindisplaymathZ
\lthtmlcheckvsize\clearpage}

{\newpage\clearpage
\lthtmlinlinemathA{tex2html_wrap_indisplay4959}%
$\displaystyle \{ i_{l,z} + i s_{l,z} : i = 0, 1, \ldots, (n_{l,z}-1) \} .$%
\lthtmlindisplaymathZ
\lthtmlcheckvsize\clearpage}

\stepcounter{section}
{\newpage\clearpage
\lthtmlinlinemathA{tex2html_wrap_inline1730}%
$l > 0$%
\lthtmlinlinemathZ
\lthtmlcheckvsize\clearpage}

{\newpage\clearpage
\lthtmlinlinemathA{tex2html_wrap_inline1732}%
$P$%
\lthtmlinlinemathZ
\lthtmlcheckvsize\clearpage}

{\newpage\clearpage
\lthtmlinlinemathA{tex2html_wrap_inline1734}%
$m > l$%
\lthtmlinlinemathZ
\lthtmlcheckvsize\clearpage}

\stepcounter{chapter}
\stepcounter{chapter}
\stepcounter{section}
\stepcounter{subsection}
{\newpage\clearpage
\lthtmlfigureA{display1807}%
\begin{display}\begin{verbatim}

int FooBar();
AVSmodule_from_desc(FooBar);\end{verbatim}
\end{display}%
\lthtmlfigureZ
\lthtmlcheckvsize\clearpage}

{\newpage\clearpage
\lthtmlfigureA{display1814}%
\begin{display}\begin{verbatim}

build install\end{verbatim}
\end{display}%
\lthtmlfigureZ
\lthtmlcheckvsize\clearpage}

\stepcounter{subsection}
{\newpage\clearpage
\lthtmlfigureA{display1835}%
\begin{display}\begin{verbatim}

file foo_bar\end{verbatim}
\end{display}%
\lthtmlfigureZ
\lthtmlcheckvsize\clearpage}

{\newpage\clearpage
\lthtmlfigureA{display1839}%
\begin{display}\begin{verbatim}

build install\end{verbatim}
\end{display}%
\lthtmlfigureZ
\lthtmlcheckvsize\clearpage}

\stepcounter{section}
\stepcounter{subsection}
{\newpage\clearpage
\lthtmlinlinemathA{tex2html_wrap_inline1966}%
$10\times10\times10$%
\lthtmlinlinemathZ
\lthtmlcheckvsize\clearpage}

\stepcounter{subsection}
\stepcounter{subsubsection}
\stepcounter{subsubsection}
{\newpage\clearpage
\lthtmlfigureA{display1892}%
\begin{display}\begin{verbatim}
vxp -proj project_dir -compile -exit [-none]\end{verbatim}
\end{display}%
\lthtmlfigureZ
\lthtmlcheckvsize\clearpage}

{\newpage\clearpage
\lthtmlfigureA{display1901}%
\begin{display}\begin{verbatim}
perl -pi.bak -e s/-lsunmath// v/templ.v\end{verbatim}
\end{display}%
\lthtmlfigureZ
\lthtmlcheckvsize\clearpage}

{\newpage\clearpage
\lthtmlfigureA{display1908}%
\begin{display}\begin{verbatim}
vxp -path "proj_dir global_express_dir" -generate -exit [-none]\end{verbatim}
\end{display}%
\lthtmlfigureZ
\lthtmlcheckvsize\clearpage}

\stepcounter{subsubsection}
\stepcounter{subsubsection}
\stepcounter{section}
\stepcounter{subsection}
\stepcounter{chapter}
\stepcounter{section}
\stepcounter{section}
\stepcounter{section}
\stepcounter{subsection}
{\newpage\clearpage
\lthtmlfigureA{display2117}%
\begin{display}\begin{verbatim}

input
output\end{verbatim}
\end{display}%
\lthtmlfigureZ
\lthtmlcheckvsize\clearpage}

\stepcounter{subsection}
\stepcounter{subsection}
\stepcounter{subsection}
\stepcounter{subsection}
\stepcounter{subsection}
\stepcounter{subsection}
\stepcounter{chapter}
\stepcounter{section}
\stepcounter{subsection}
{\newpage\clearpage
\lthtmlfigureA{display2315}%
\begin{display}\begin{verbatim}

typedef struct
{
   double   X,  Y,  Z;    /* Bottom-lower-left corner in real-space */
   int      NX, NY, NZ;   /* Size in each coordinate direction */
   double   DX, DY, DZ;   /* Spacing in each coordinate direction */

} Background;\end{verbatim}
\end{display}%
\lthtmlfigureZ
\lthtmlcheckvsize\clearpage}

{\newpage\clearpage
\lthtmlfigureA{display2321}%
\begin{display}\begin{verbatim}

typedef struct
{
   int  ix, iy, iz;      /* Bottom-lower-left corner in index-space */
   int  nx, ny, nz;      /* Size */
   int  sx, sy, sz;      /* Striding factors */
   int  rx, ry, rz;      /* Refinement over the background grid */
   int  level;           /* Refinement level = rx + ry + rz */

   int  process;         /* Process containing this subgrid */

} Subregion;\end{verbatim}
\end{display}%
\lthtmlfigureZ
\lthtmlcheckvsize\clearpage}

{\newpage\clearpage
\lthtmlfigureA{display2327}%
\begin{display}\begin{verbatim}

typedef struct
{
   Subregion  **subregions;   /* Array of pointers to subregions */
   int          size;         /* Size of subgregion array */

} SubregionArray;\end{verbatim}
\end{display}%
\lthtmlfigureZ
\lthtmlcheckvsize\clearpage}

{\newpage\clearpage
\lthtmlfigureA{display2331}%
\begin{display}\begin{verbatim}

typedef struct
{
   SubregionArray  **subregion_arrays;   /* Array of pointers to
                                          * subregion arrays */
   int               size;               /* Size of region */

} Region;\end{verbatim}
\end{display}%
\lthtmlfigureZ
\lthtmlcheckvsize\clearpage}

{\newpage\clearpage
\lthtmlfigureA{display2335}%
\begin{display}\begin{verbatim}

typedef Subregion  Subgrid;\end{verbatim}
\end{display}%
\lthtmlfigureZ
\lthtmlcheckvsize\clearpage}

{\newpage\clearpage
\lthtmlfigureA{display2339}%
\begin{display}\begin{verbatim}

typedef SubregionArray  SubgridArray;\end{verbatim}
\end{display}%
\lthtmlfigureZ
\lthtmlcheckvsize\clearpage}

{\newpage\clearpage
\lthtmlfigureA{display2343}%
\begin{display}\begin{verbatim}

typedef struct
{
   SubgridArray  *subgrids;     /* Array of subgrids in this process */

   SubgridArray  *all_subgrids; /* Array of all subgrids in the grid */

   int            size;         /* Total number of grid points */

   ComputePkg   **compute_pkgs;

} Grid;\end{verbatim}
\end{display}%
\lthtmlfigureZ
\lthtmlcheckvsize\clearpage}

\stepcounter{subsection}
{\newpage\clearpage
\lthtmlfigureA{display2358}%
\begin{display}\begin{verbatim}

Background      *background;
Subregion       *subregion;
SubregionArray  *subregion_array;
Region          *region;
Subgrid         *subgrid;
SubgridArray    *subgrid_array;
Grid            *grid;
int              i;\end{verbatim}
\end{display}%
\lthtmlfigureZ
\lthtmlcheckvsize\clearpage}

{\newpage\clearpage
\lthtmlfigureA{display2362}%
\begin{display}\begin{verbatim}

BackgroundX(background)   background -> X
BackgroundY(background)   background -> Y
BackgroundZ(background)   background -> Z
BackgroundNX(background)  background -> NX
BackgroundNY(background)  background -> NY
BackgroundNZ(background)  background -> NZ
BackgroundDX(background)  background -> DX
BackgroundDY(background)  background -> DY
BackgroundDZ(background)  background -> DZ\end{verbatim}
\end{display}%
\lthtmlfigureZ
\lthtmlcheckvsize\clearpage}

{\newpage\clearpage
\lthtmlfigureA{display2366}%
\begin{display}\begin{verbatim}

SubregionIX(subregion)      subregion -> ix
SubregionIY(subregion)      subregion -> iy
SubregionIZ(subregion)      subregion -> iz
SubregionNX(subregion)      subregion -> nx
SubregionNY(subregion)      subregion -> ny
SubregionNZ(subregion)      subregion -> nz
SubregionSX(subregion)      subregion -> sx
SubregionSY(subregion)      subregion -> sy
SubregionSZ(subregion)      subregion -> sz
SubregionRX(subregion)      subregion -> rx
SubregionRY(subregion)      subregion -> ry
SubregionRZ(subregion)      subregion -> rz
SubregionLevel(subregion)   subregion -> level
SubregionProcess(subregion) subregion -> process\end{verbatim}
\end{display}%
\lthtmlfigureZ
\lthtmlcheckvsize\clearpage}

{\newpage\clearpage
\lthtmlfigureA{display2370}%
\begin{display}\begin{verbatim}

SubregionArraySubregion(subregion_array, i)  subregion_array -> subregions[i]
SubregionArraySize(subregion_array)          subregion_array -> size\end{verbatim}
\end{display}%
\lthtmlfigureZ
\lthtmlcheckvsize\clearpage}

{\newpage\clearpage
\lthtmlfigureA{display2374}%
\begin{display}\begin{verbatim}

RegionSubregionArray(region, i)  region -> subregion_arrays[i]
RegionSize(region)               region -> size\end{verbatim}
\end{display}%
\lthtmlfigureZ
\lthtmlcheckvsize\clearpage}

{\newpage\clearpage
\lthtmlfigureA{display2378}%
\begin{display}\begin{verbatim}

SubgridIX(subgrid)       SubregionIX(subgrid)
SubgridIY(subgrid)       SubregionIY(subgrid)
SubgridIZ(subgrid)       SubregionIZ(subgrid)
SubgridNX(subgrid)       SubregionNX(subgrid)
SubgridNY(subgrid)       SubregionNY(subgrid)
SubgridNZ(subgrid)       SubregionNZ(subgrid)
SubgridRX(subgrid)       SubregionRX(subgrid)
SubgridRY(subgrid)       SubregionRY(subgrid)
SubgridRZ(subgrid)       SubregionRZ(subgrid)
SubgridLevel(subgrid)    SubregionLevel(subgrid)
SubgridProcess(subgrid)  SubregionProcess(subgrid)\end{verbatim}
\end{display}%
\lthtmlfigureZ
\lthtmlcheckvsize\clearpage}

{\newpage\clearpage
\lthtmlfigureA{display2382}%
\begin{display}\begin{verbatim}

SubgridArraySubgrid(subgrid_array, i)  SubregionArraySubregion(subgrid_array, i)
SubgridArraySize(subgrid_array)        SubregionArraySize(subgrid_array)\end{verbatim}
\end{display}%
\lthtmlfigureZ
\lthtmlcheckvsize\clearpage}

{\newpage\clearpage
\lthtmlfigureA{display2386}%
\begin{display}\begin{verbatim}

GridSubgrids(grid)       grid -> subgrids
GridAllSubgrids(grid)    grid -> all_subgrids
GridSize(grid)           grid -> size
GridComputePkgs(grid)    grid -> compute_pkgs
GridComputePkg(grid, i)  grid -> compute_pkgs[i]
GridSubgrid(grid, i)     SubgridArraySubgrid(GridSubgrids(grid), i)
GridNumSubgrids(grid)    SubgridArraySize(GridSubgrids(grid))\end{verbatim}
\end{display}%
\lthtmlfigureZ
\lthtmlcheckvsize\clearpage}

\stepcounter{subsection}
{\newpage\clearpage
\lthtmlfigureA{deftypefn2393}%
\begin{deftypefn}{Function}{Background *}{ReadBackground}(amps_File \var{file})
\par
\DESCRIPTION
Reads in a background description from \var{file}, then
creates and returns a new \code{Background} structure.
\par
\SEEALSO
\vref{Grid Structures}{Grid Structures}\\
\vref{Gridding}{Gridding}
\par
\end{deftypefn}%
\lthtmlfigureZ
\lthtmlcheckvsize\clearpage}

\stepcounter{subsection}
{\newpage\clearpage
\lthtmlfigureA{deftypefn2408}%
\begin{deftypefn}{Function}{Subregion *}{NewSubregion}(int \var{ix}, int \var{iy}, int \var{iz}, int \var{nx}, int \var{ny}, int \var{nz}, int \var{sx}, int \var{sy}, int \var{sz}, int \var{rx}, int \var{ry}, int \var{rz}, int \var{process})
\par
\DESCRIPTION
Creates and returns a pointer to a new \code{Subregion} structure.
\par
\SEEALSO
\vref{Grid Structures}{Grid Structures}\\
\vref{Gridding}{Gridding}
\par
\end{deftypefn}%
\lthtmlfigureZ
\lthtmlcheckvsize\clearpage}

\stepcounter{subsection}
{\newpage\clearpage
\lthtmlfigureA{deftypefn2434}%
\begin{deftypefn}{Function}{SubregionArray *}{NewSubregionArray}()
\par
\DESCRIPTION
Creates and returns a pointer to a new \code{SubregionArray} structure
of size 0.
\par
\SEEALSO
\vref{Grid Structures}{Grid Structures}\\
\vref{AppendSubregion}{AppendSubregion}\\
\vref{AppendSubregionArray}{AppendSubregionArray}\\
\vref{Gridding}{Gridding}
\par
\end{deftypefn}%
\lthtmlfigureZ
\lthtmlcheckvsize\clearpage}

\stepcounter{subsection}
{\newpage\clearpage
\lthtmlfigureA{deftypefn2451}%
\begin{deftypefn}{Function}{Region *}{NewRegion}(int \var{size})
\par
\DESCRIPTION
Creates and returns a pointer to a new \code{Region} structure
containing \var{size} pointers to new \code{SubregionArray} structures.
\par
\SEEALSO
\vref{Grid Structures}{Grid Structures}\\
\vref{NewSubregionArray}{NewSubregionArray}\\
\vref{AppendSubregion}{AppendSubregion}\\
\vref{AppendSubregionArray}{AppendSubregionArray}\\
\vref{Gridding}{Gridding}
\par
\end{deftypefn}%
\lthtmlfigureZ
\lthtmlcheckvsize\clearpage}

\stepcounter{subsection}
{\newpage\clearpage
\lthtmlfigureA{deftypefn2473}%
\begin{deftypefn}
% latex2html id marker 2473
{Function}{Subgrid *}{NewSubgrid}(int \var{ix}, int \var{iy}, int \var{iz}, int \var{nx}, int \var{ny}, int \var{nz}, int \var{rx}, int \var{ry}, int \var{rz}, int \var{process})
\par
\DESCRIPTION
Creates and returns a pointer to a new \code{Subgrid} structure.
\par
\NOTES
This is currently a macro (\ref{NewSubregion}).
\par
\SEEALSO
\vref{Grid Structures}{Grid Structures}\\
\vref{Gridding}{Gridding}
\par
\end{deftypefn}%
\lthtmlfigureZ
\lthtmlcheckvsize\clearpage}

\stepcounter{subsection}
{\newpage\clearpage
\lthtmlfigureA{deftypefn2497}%
\begin{deftypefn}{Function}{SubgridArray *}{NewSubgridArray}()
\par
\DESCRIPTION
Creates and returns a pointer to a new \code{SubgridArray} structure
of size 0.
\par
\SEEALSO
\vref{Grid Structures}{Grid Structures}\\
\vref{AppendSubgrid}{AppendSubgrid}\\
\vref{AppendSubgridArray}{AppendSubgridArray}\\
\vref{Gridding}{Gridding}
\par
\end{deftypefn}%
\lthtmlfigureZ
\lthtmlcheckvsize\clearpage}

\stepcounter{subsection}
{\newpage\clearpage
\lthtmlfigureA{deftypefn2514}%
\begin{deftypefn}{Function}{Grid *}{NewGrid}({SubgridArray *} \var{subgrids}, {SubgridArray *} \var{all\_subgrids})
\par
\DESCRIPTION
Creates and returns a pointer to a new \code{Grid} structure.
The \code{size} component is computed and the \code{compute_pkgs}
component is set to \code{NULL}.
\par
\SEEALSO
\vref{Grid Structures}{Grid Structures}\\
\vref{Gridding}{Gridding}
\par
\end{deftypefn}%
\lthtmlfigureZ
\lthtmlcheckvsize\clearpage}

\stepcounter{subsection}
{\newpage\clearpage
\lthtmlfigureA{deftypefn2534}%
\begin{deftypefn}{Function}{void}{FreeBackground}({Background *} \var{background})
\par
\DESCRIPTION
Frees a \code{Background} structure.
\par
\SEEALSO
\vref{Grid Structures}{Grid Structures}\\
\vref{Gridding}{Gridding}
\par
\end{deftypefn}%
\lthtmlfigureZ
\lthtmlcheckvsize\clearpage}

\stepcounter{subsection}
{\newpage\clearpage
\lthtmlfigureA{deftypefn2549}%
\begin{deftypefn}{Function}{void}{FreeSubregion}({Subregion *} \var{subregion})
\par
\DESCRIPTION
Frees a \code{Subregion} structure.
\par
\SEEALSO
\vref{Grid Structures}{Grid Structures}\\
\vref{Gridding}{Gridding}
\par
\end{deftypefn}%
\lthtmlfigureZ
\lthtmlcheckvsize\clearpage}

\stepcounter{subsection}
{\newpage\clearpage
\lthtmlfigureA{deftypefn2564}%
\begin{deftypefn}{Function}{void}{FreeSubregionArray}({SubregionArray *} \var{subregion\_array})
\par
\DESCRIPTION
Frees a \code{SubregionArray} structure, and all \code{Subregion}
structures pointed to by it.
Often times one may want to free up the \code{SubregionArray} structure
without freeing its \code{Subregion} structures.
This is easily achieved as in the following example.
\par
\EXAMPLE
\mbox{}
\begin{display}\begin{verbatim}

SubregionArray  *subregion_array;

   ...

SubregionArraySize(subregion_array) = 0;
FreeSubregionArray(subregion_array);\end{verbatim}
\end{display}
\par
\SEEALSO
\vref{Grid Structures}{Grid Structures}\\
\vref{FreeSubregion}{FreeSubregion}\\
\vref{Gridding}{Gridding}
\par
\end{deftypefn}%
\lthtmlfigureZ
\lthtmlcheckvsize\clearpage}

\stepcounter{subsection}
{\newpage\clearpage
\lthtmlfigureA{deftypefn2589}%
\begin{deftypefn}{Function}{void}{FreeRegion}({Region *} \var{region})
\par
\DESCRIPTION
Frees a \code{Region} structure and all \code{SubregionArray}
structures pointed to by it.
Sometimes one may want to free up the \code{Region} structure
without freeing its \code{SubregionArray} structures.
This is easily achieved as in the following example.
\par
\EXAMPLE
\mbox{}
\begin{display}\begin{verbatim}

Region  *region;

   ...

RegionSize(region) = 0;
FreeRegion(region);\end{verbatim}
\end{display}
\par
\SEEALSO
\vref{Grid Structures}{Grid Structures}\\
\vref{FreeSubregionArray}{FreeSubregionArray}\\
\vref{Gridding}{Gridding}
\par
\end{deftypefn}%
\lthtmlfigureZ
\lthtmlcheckvsize\clearpage}

\stepcounter{subsection}
{\newpage\clearpage
\lthtmlfigureA{deftypefn2614}%
\begin{deftypefn}
% latex2html id marker 2614
{Function}{void}{FreeSubgrid}({Subgrid *} \var{subgrid})
\par
\DESCRIPTION
Frees a \code{Subgrid} structure.
\par
\NOTES
This is currently a macro (\ref{FreeSubregion}).
\par
\SEEALSO
\vref{Grid Structures}{Grid Structures}\\
\vref{Gridding}{Gridding}
\par
\end{deftypefn}%
\lthtmlfigureZ
\lthtmlcheckvsize\clearpage}

\stepcounter{subsection}
{\newpage\clearpage
\lthtmlfigureA{deftypefn2630}%
\begin{deftypefn}
% latex2html id marker 2630
{Function}{void}{FreeSubgridArray}({SubgridArray *} \var{subgrid\_array})
\par
\DESCRIPTION
Frees a \code{SubgridArray} structure, and all \code{Subgrid}
structures pointed to by it.
Often times one may want to free up the \code{SubgridArray} structure
without freeing its \code{Subgrid} structures.
This is easily achieved as in the following example.
\par
\EXAMPLE
\mbox{}
\begin{display}\begin{verbatim}

SubgridArray  *subgrid_array;

   ...

SubgridArraySize(subgrid_array) = 0;
FreeSubgridArray(subgrid_array);\end{verbatim}
\end{display}
\par
\NOTES
This is currently a macro (\ref{FreeSubregionArray}).
\par
\SEEALSO
\vref{Grid Structures}{Grid Structures}\\
\vref{FreeSubgrid}{FreeSubgrid}\\
\vref{Gridding}{Gridding}
\par
\end{deftypefn}%
\lthtmlfigureZ
\lthtmlcheckvsize\clearpage}

\stepcounter{subsection}
{\newpage\clearpage
\lthtmlfigureA{deftypefn2656}%
\begin{deftypefn}{Function}{void}{FreeGrid}({Grid *} \var{grid})
\par
\DESCRIPTION
Frees a \code{Grid} structure and all structures contained therein.
\par
\SEEALSO
\vref{Grid Structures}{Grid Structures}\\
\vref{FreeSubgridArray}{FreeSubgridArray}\\
\vref{Gridding}{Gridding}
\par
\end{deftypefn}%
\lthtmlfigureZ
\lthtmlcheckvsize\clearpage}

\stepcounter{subsection}
{\newpage\clearpage
\lthtmlfigureA{defmac2673}%
\begin{defmac} ForSubregionI (int \var{i}, {SubregionArray *} \var{subregion\_array})
\par
\DESCRIPTION
Loops over all subregion indices in \var{subregion\_array}, setting
\var{i} to each index.
\par
\EXAMPLE
\mbox{}
\begin{display}\begin{verbatim}

SubregionArray  *subregion_array;
Subregion       *subregion;
int              i;

ForSubregionI(i, subregion_array)
{
   subregion = SubregionArraySubregion(subregion_array, i);

   ...
}\end{verbatim}
\end{display}
\par
\SEEALSO
\vref{Grid Structures}{Grid Structures}\\
\vref{Gridding}{Gridding}
\par
\end{defmac}%
\lthtmlfigureZ
\lthtmlcheckvsize\clearpage}

\stepcounter{subsection}
{\newpage\clearpage
\lthtmlfigureA{defmac2692}%
\begin{defmac} ForSubregionArrayI (int \var{i}, {Region *} \var{region})
\par
\DESCRIPTION
Loops over all subregion-array indices in \var{region}, setting
\var{i} to each index.
\par
\EXAMPLE
\mbox{}
\begin{display}\begin{verbatim}

Region          *region;
SubregionArray  *subregion_array;
int              i;

ForSubregionArrayI(i, region)
{
   subregion_array = RegionSubregionArray(region, i);

   ...
}\end{verbatim}
\end{display}
\par
\SEEALSO
\vref{Grid Structures}{Grid Structures}\\
\vref{Gridding}{Gridding}
\par
\end{defmac}%
\lthtmlfigureZ
\lthtmlcheckvsize\clearpage}

\stepcounter{subsection}
{\newpage\clearpage
\lthtmlfigureA{defmac2711}%
\begin{defmac} ForSubgridI (int \var{i}, {SubgridArray *} \var{subgrid\_array})
\par
\DESCRIPTION
Loops over all subgrid indices in \var{subgrid\_array}, setting
\var{i} to each index.
\par
\EXAMPLE
\mbox{}
\begin{display}\begin{verbatim}

SubgridArray *subgrid_array;
Subgrid      *subgrid;
int           i;

ForSubgridI(i, subgrid_array)
{
   subgrid = SubgridArraySubgrid(subgrid_array, i);

   ...
}\end{verbatim}
\end{display}
\par
\SEEALSO
\vref{Grid Structures}{Grid Structures}\\
\vref{Gridding}{Gridding}
\par
\end{defmac}%
\lthtmlfigureZ
\lthtmlcheckvsize\clearpage}

\stepcounter{subsection}
{\newpage\clearpage
\lthtmlfigureA{deftypefn2730}%
\begin{deftypefn}{Function}{double}{SubregionX}({Subregion *} \var{subregion})
\end{deftypefn}%
\lthtmlfigureZ
\lthtmlcheckvsize\clearpage}

{\newpage\clearpage
\lthtmlfigureA{deftypefn2737}%
\begin{deftypefn}{Function}{double}{SubregionY}({Subregion *} \var{subregion})
\end{deftypefn}%
\lthtmlfigureZ
\lthtmlcheckvsize\clearpage}

{\newpage\clearpage
\lthtmlfigureA{deftypefn2744}%
\begin{deftypefn}{Function}{double}{SubregionZ}({Subregion *} \var{subregion})
\par
\DESCRIPTION
Returns the real-space coordinates corresponding to the
\code{ix}, \code{iy}, and \code{iz} components of \var{subregion}.
\par
\NOTES
This is actually a macro.
\par
\SEEALSO
\vref{Grid Structures}{Grid Structures}\\
\vref{Gridding}{Gridding}
\par
\end{deftypefn}%
\lthtmlfigureZ
\lthtmlcheckvsize\clearpage}

\stepcounter{subsection}
{\newpage\clearpage
\lthtmlfigureA{deftypefn2762}%
\begin{deftypefn}{Function}{double}{SubgridX}({Subgrid *} \var{subgrid})
\end{deftypefn}%
\lthtmlfigureZ
\lthtmlcheckvsize\clearpage}

{\newpage\clearpage
\lthtmlfigureA{deftypefn2769}%
\begin{deftypefn}{Function}{double}{SubgridY}({Subgrid *} \var{subgrid})
\end{deftypefn}%
\lthtmlfigureZ
\lthtmlcheckvsize\clearpage}

{\newpage\clearpage
\lthtmlfigureA{deftypefn2776}%
\begin{deftypefn}{Function}{double}{SubgridZ}({Subgrid *} \var{subgrid})
\par
\DESCRIPTION
Returns the real-space coordinates corresponding to the
\code{ix}, \code{iy}, and \code{iz} components of \var{subgrid}.
\par
\NOTES
This is actually a macro.
\par
\SEEALSO
\vref{Grid Structures}{Grid Structures}\\
\vref{Gridding}{Gridding}
\par
\end{deftypefn}%
\lthtmlfigureZ
\lthtmlcheckvsize\clearpage}

\stepcounter{subsection}
{\newpage\clearpage
\lthtmlfigureA{deftypefn2794}%
\begin{deftypefn}{Function}{Subregion *}{DuplicateSubregion}({Subregion *} \var{subregion})
\par
\DESCRIPTION
Returns a duplicate of \var{subregion}.
\par
\SEEALSO
\vref{Gridding}{Gridding}
\par
\end{deftypefn}%
\lthtmlfigureZ
\lthtmlcheckvsize\clearpage}

\stepcounter{subsection}
{\newpage\clearpage
\lthtmlfigureA{deftypefn2807}%
\begin{deftypefn}{Function}{void}{AppendSubregion}({Subregion *} \var{subregion}, {SubregionArray *} \var{subregion\_array})
\par
\DESCRIPTION
Appends \var{subregion} to \var{subregion\_array}.
\par
\NOTES
{\bf Important:}
No new \code{Subregion} structure is created; only the pointer
is copied into \var{subregion\_array}.
If one is not careful, it is easy to free \code{Subregion} structures
accidentally.
\par
\SEEALSO
\vref{Gridding}{Gridding}
\par
\end{deftypefn}%
\lthtmlfigureZ
\lthtmlcheckvsize\clearpage}

\stepcounter{subsection}
{\newpage\clearpage
\lthtmlfigureA{deftypefn2827}%
\begin{deftypefn}{Function}{void}{AppendSubregionArray}({SubregionArray *} \var{subregion\_array\_0}, {SubregionArray *} \var{subregion\_array\_1})
\par
\DESCRIPTION
Appends \var{subregion\_array\_0} to \var{subregion\_array\_1}.
\par
\NOTES
{\bf Important:}
No new \code{Subregion} structures are created; only pointers
are copied into the \var{subregion\_array\_1}.
If one is not careful, it is easy to free \code{Subregion} structures
accidentally.
\par
\SEEALSO
\vref{Gridding}{Gridding}
\par
\end{deftypefn}%
\lthtmlfigureZ
\lthtmlcheckvsize\clearpage}

\stepcounter{subsection}
{\newpage\clearpage
\lthtmlfigureA{deftypefn2847}%
\begin{deftypefn}
% latex2html id marker 2847
{Function}{Subgrid *}{DuplicateSubgrid}({Subgrid *} \var{subgrid})
\par
\DESCRIPTION
Returns a duplicate of \var{subgrid}.
\par
\NOTES
This is currently a macro (\ref{DuplicateSubregion}).
\par
\SEEALSO
\vref{Gridding}{Gridding}
\par
\end{deftypefn}%
\lthtmlfigureZ
\lthtmlcheckvsize\clearpage}

\stepcounter{subsection}
{\newpage\clearpage
\lthtmlfigureA{deftypefn2861}%
\begin{deftypefn}
% latex2html id marker 2861
{Function}{void}{AppendSubgrid}({Subgrid *} \var{subgrid}, {SubgridArray *} \var{subgrid\_array})
\par
\DESCRIPTION
Appends \var{subgrid} to \var{subgrid\_array}.
\par
\NOTES
This is currently a macro (\ref{AppendSubregion}).
\par
\SEEALSO
\vref{Gridding}{Gridding}
\par
\end{deftypefn}%
\lthtmlfigureZ
\lthtmlcheckvsize\clearpage}

\stepcounter{subsection}
{\newpage\clearpage
\lthtmlfigureA{deftypefn2878}%
\begin{deftypefn}
% latex2html id marker 2878
{Function}{void}{AppendSubgridArray}({SubgridArray *} \var{subgrid\_array\_0}, {SubgridArray *} \var{subgrid\_array\_1})
\par
\DESCRIPTION
Appends \var{subgrid\_array\_0} to \var{subgrid\_array\_1}.
\par
\NOTES
This is currently a macro (\ref{AppendSubregionArray}).
\par
\SEEALSO
\vref{Gridding}{Gridding}
\par
\end{deftypefn}%
\lthtmlfigureZ
\lthtmlcheckvsize\clearpage}

\stepcounter{subsection}
{\newpage\clearpage
\lthtmlfigureA{deftypefn2895}%
\begin{deftypefn}{Function}{Subregion *}{ConvertToSubregion}({Subgrid *} \var{subgrid})
\par
\DESCRIPTION
Converts \var{subgrid} to a \code{Subregion *} and returns it.
\par
\NOTES
This is currently a macro which simply does a cast.
\par
\SEEALSO
\vref{ConvertToSubgrid}{ConvertToSubgrid}\\
\vref{Gridding}{Gridding}
\par
\end{deftypefn}%
\lthtmlfigureZ
\lthtmlcheckvsize\clearpage}

\stepcounter{subsection}
{\newpage\clearpage
\lthtmlfigureA{deftypefn2911}%
\begin{deftypefn}{Function}{Subgrid *}{ConvertToSubgrid}({Subregion *} \var{subregion})
\par
\DESCRIPTION
Converts \var{subregion} to a \code{Subgrid *} if possible,
and returns it.
\var{subregion} is not duplicated, it is modified.
If a conversion cannot be made, \code{NULL} is returned.
\par
\SEEALSO
\vref{ConvertToSubregion}{ConvertToSubregion}\\
\vref{Gridding}{Gridding}
\par
\end{deftypefn}%
\lthtmlfigureZ
\lthtmlcheckvsize\clearpage}

\stepcounter{subsection}
{\newpage\clearpage
\lthtmlfigureA{deftypefn2929}%
\begin{deftypefn}{Function}{Subregion *}{ProjectSubgrid}({Subgrid *} \var{subgrid}, int \var{sx}, int \var{sy}, int \var{sz}, int \var{ix}, int \var{iy}, int \var{iz})
\par
\DESCRIPTION
Projects \var{subgrid} onto an {\em index-region}.
The index-region is the collection of index-space indices with
strides \var{sx}, \var{sy}, \var{sz}, and containing
the indices \var{ix}, \var{iy}, \var{iz}.
The base index-space is determined by the \var{subgrid}.
\var{subgrid} is not modified.
\par
\SEEALSO
\vref{Gridding}{Gridding}
\par
\end{deftypefn}%
\lthtmlfigureZ
\lthtmlcheckvsize\clearpage}

\stepcounter{subsection}
{\newpage\clearpage
\lthtmlfigureA{deftypefn2957}%
\begin{deftypefn}{Function}{Subgrid *}{ExtractSubgrid}(int \var{rx}, int \var{ry}, int \var{rz}, {Subgrid *} \var{subgrid})
\par
\DESCRIPTION
Returns, if possible, the largest subgrid with resolution
\var{rx}, \var{ry}, \var{rz},
contained in \var{subgrid}
\par
\NOTES
This routine is currently used in only one place and is very similar
to \code{ProjectSubgrid}, so can probably be eliminated at some point.
\par
\SEEALSO
\vref{ProjectSubgrid}{ProjectSubgrid}\\
\vref{Gridding}{Gridding}
\par
\end{deftypefn}%
\lthtmlfigureZ
\lthtmlcheckvsize\clearpage}

\stepcounter{subsection}
{\newpage\clearpage
\lthtmlfigureA{deftypefn2979}%
\begin{deftypefn}
% latex2html id marker 2979
{Function}{Subgrid *}{IntersectSubgrids}({Subgrid *} \var{subgrid1}, {Subgrid *} \var{subgrid2})
\par
\DESCRIPTION
Returns the intersection of \var{subgrid1} and \var{subgrid2}.
Returns \code{NULL} if none.
\par
\NOTES
Assumes the index spaces for the two subgrids are legally nested
(\ref{Gridding}.
\par
\SEEALSO
\vref{Gridding}{Gridding}
\par
\end{deftypefn}%
\lthtmlfigureZ
\lthtmlcheckvsize\clearpage}

\stepcounter{subsection}
{\newpage\clearpage
\lthtmlfigureA{deftypefn2997}%
\begin{deftypefn}{Function}{SubgridArray *}{SubtractSubgrids}({Subgrid *} \var{subgrid1}, {Subgrid *} \var{subgrid2})
\par
\DESCRIPTION
Subtracts the intersection of \var{subgrid1} and \var{subgrid2}
from \var{subgrid1}.
\par
\SEEALSO
\vref{Gridding}{Gridding}
\par
\end{deftypefn}%
\lthtmlfigureZ
\lthtmlcheckvsize\clearpage}

\stepcounter{subsection}
{\newpage\clearpage
\lthtmlfigureA{deftypefn3014}%
\begin{deftypefn}{Function}{SubgridArray *}{UnionSubgridArray}({SubgridArray *} \var{subgrid\_array})
\par
\DESCRIPTION
Unions the subgrids in \var{subgrid\_array}.
This involves removing any duplication of grid points, resulting in
a subgrid-array of disjoint subgrids.
The routine also tries to agglomerate subgrids into bigger subgrids.
\par
\NOTES
{\bf Important:} This routine ignores process numbers, and currently
sets all process values in the returned union to 0.
\par
\SEEALSO
\vref{Gridding}{Gridding}
\par
\end{deftypefn}%
\lthtmlfigureZ
\lthtmlcheckvsize\clearpage}


\end{document}
