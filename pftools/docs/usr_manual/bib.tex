\begin{thebibliography}{999}

\bibitem{Ashby-Falgout90}
Ashby, S.F. and R.~D. Falgout (1996).
\newblock A parallel multigrid preconditioned conjugate gradient algorithm for
  groundwater flow simulations.
\newblock {\em Nuclear Science and Engineering}, {\bf 124}:145--159.

\bibitem{AM10}
Atchley, A. and Maxwell, R.M. (2010). Influences of subsurface heterogeneity and vegetation cover on soil moisture, surface temperature, and evapotranspiration at hillslope scales. {\em Hydrogeology Journal} doi:10.1007/s10040-010-0690-1.

\bibitem{Dai03}
Dai, Y., X. Zeng, R.E. Dickinson, I. Baker, G.B. Bonan, M.G. Bosilovich, A.S. Denning, P.A. Dirmeyer, P.R., G. Niu, K.W. Oleson, C.A. Schlosser and Z.L. Yang (2003). The common land model. {\em The Bulletin of the American Meteorological Society} {\bf 84}(8):1013--1023.

\bibitem{DMC10}
Daniels, M.H., Maxwell, R.M., Chow, F.K. (2010). An algorithm for flow direction enforcement using subgrid-scale stream location data, {\em Journal of Hydrologic Engineering} doi:10.1061/(ASCE)HE.1943-5584.0000340.

\bibitem{dBRM08}
de Barros, F.P.J., Rubin, Y. and Maxwell, R.M. (2009). The concept of comparative information yield curves and their application to risk-based site characterization. {\em Water Resources Research} doi:10.1029/2008WR007324, 45, W06401.

\bibitem{EW96}
Eisenstat, S.C. and H.F. Walker (1996).
\newblock Choosing the forcing terms in an inexact newton method.
\newblock {\em SIAM J. Sci. Comput.}, {\bf 17}(1):16--32.

\bibitem{FM10}
Ferguson, I.M. and Maxwell, R.M. (2010). The role of groundwater in watershed response and land surface feedbacks under climate change. {\em Water Resources Research} 46, W00F02, doi:10.1029/2009WR008616.

\bibitem{FWP95}
Forsyth, P.A. Wu, Y.S. and Pruess, K. (1995).
\newblock Robust Numerical Methods for Saturated-Unsaturated Flow with Dry Initial Conditions.
\newblock {\em Advances in Water Resources}, {\bf 17}:25--38.

\bibitem{FFKM09}
Frei, S., Fleckenstein, J.H., Kollet, S.J. and Maxwell, R.M. (2009). Patterns and dynamics of river-aquifer exchange with variably-saturated flow using a fully-coupled model. {\em Journal of Hydrology} 375(3-4), 383-393, doi:10.1016/j.jhydrol.2009.06.038.

\bibitem{Haverkamp-Vauclin81}
Haverkamp, R. and M.~Vauclin (1981).
\newblock A comparative study of three forms of the {R}ichard equation used for
  predicting one-dimensional infiltration in unsaturated soil.
\newblock {\em Soil Sci. Soc. of Am. J.}, {\bf 45}:13--20.


\bibitem{Jones-Woodward01}
Jones, J.E. and C.~S. Woodward (2001).
\newblock Newton-krylov-multigrid solvers for large-scale, highly heterogeneous, variably saturated flow problems.
\newblock {\em Advances in Water Resources}, {\bf 24}:763--774.

\bibitem{KRM10}
Kollat, J.B., Reed P.M. and Maxwell, R.M. (2010). Many-Objective Groundwater Monitoring Network Design Using Bias-Aware Ensemble Kalman Filtering, Evolutionary Optimization, and Visual Analytics. {\em Water Resources Research},doi:10.1029/2010WR009194.

\bibitem{KM06}
Kollet, S.~J. and R.~M. Maxwell, (2006). Integrated
surface-groundwater flow
  modeling: A free-surface overland flow boundary condition in a parallel
  groundwater flow model. {\em Advances in Water Resources}, {\bf 29}:945--958 .

\bibitem{KM08a}
Kollet, S.J. and Maxwell, R.M. (2008). Capturing the influence of groundwater dynamics on land surface processes using an integrated, distributed watershed model, { \em Water Resources Research},{\bf 44}: W02402.

\bibitem{KM08b}
Kollet, S.J. and Maxwell, R.M. (2008). Demonstrating fractal scaling of baseflow residence time distributions using a fully-coupled groundwater and land surface model. {\em Geophysical Research Letters}, {\bf 35}, L07402. 

\bibitem{KMWSVVS10}
Kollet, S.J., Maxwell, R.M., Woodward, C.S., Smith, S.G., Vanderborght, J., Vereecken, H., and Simmer, C. (2010). Proof-of-concept of regional scale hydrologic simulations at hydrologic resolution utilizing massively parallel computer resources. {\em Water Resources Research}, 46, W04201, doi:10.1029/2009WR008730.

\bibitem{KCSMMB09}
Kollet, S.J., Cvijanovic, I., Sch�ttemeyer, D., Maxwell, R.M., Moene, A.F. and Bayer P (2009). The influence of rain sensible heat, subsurface heat convection and the lower temperature boundary condition on the energy balance at the land surface. {\em Vadose Zone Journal}, doi:10.2136/vzj2009.0005.

\bibitem{MWT03} Maxwell, R.M., C. Welty, and A.F.B. Tompson (2003).
Streamline-based simulation of virus transport resulting from long term
artificial recharge in a heterogeneous aquifer {\em Advances in Water
Resources}, {\bf 22}(3):203--221.

\bibitem{MM05}
Maxwell, R.M. and N.L. Miller (2005). Development of a coupled land surface and groundwater model.  {\em Journal of Hydrometeorology}, {\bf 6}(3):233--247.

\bibitem{MCT00}
Maxwell, R.M., S.F. Carle and A.F.B. Tompson (2000). Risk-Based Management of Contaminated Groundwater: The Role of Geologic Heterogeneity, Exposure and Cancer Risk in Determining the Performance of Aquifer Remediation, In {\em Proceedings of Computational Methods in Water Resources XII}, Balkema, 533--539.

\bibitem{MWH07} Maxwell, R.M., C. Welty and R.W. Harvey (2007). Revisiting the Cape Cod Bacteria Injection Experiment Using a Stochastic Modeling Approach, {\em Environmental Science and Technology}, { \bf 41}(15):5548--5558.

\bibitem{MCK07}
Maxwell, R.M., F.K. Chow and S.J. Kollet (2007). The groundwater-land-surface-atmosphere connection: soil moisture effects on the atmospheric boundary layer in fully-coupled simulations. {\em Advances in Water Resources}, {\bf 30}(12):2447--2466 .

\bibitem {MCT08}
Maxwell, R.M., Carle, S.F. and Tompson, A.F.B. (2008).
Contamination, Risk, and Heterogeneity: On the Effectiveness of Aquifer Remediation. {\em Environmental Geology}, {\bf 54}:1771--1786.

\bibitem{MK08a}
Maxwell, R.M. and Kollet, S.J. (2008). Quantifying the effects of three-dimensional subsurface heterogeneity on Hortonian runoff processes using a coupled numerical, stochastic approach. {\em Advances in Water Resources} {\bf 31}(5): 807--817. 

\bibitem{MK08b}
Maxwell, R.M. and Kollet, S.J. (2008) Interdependence of groundwater dynamics and land-energy feedbacks under climate change. {\em Nature Geoscience} {\bf 1}(10): 665--669.

\bibitem{MTK09}
Maxwell, R.M., Tompson, A.F.B. and Kollet, S.J. (2009) A Serendipitous, Long-Term Infiltration Experiment: Water and Tritium Circulation Beneath the CAMBRIC Trench at the Nevada Test Site. {\em Journal of Contaminant Hydrology} 108(1-2) 12-28, doi:10.1016/j.jconhyd.2009.05.002.

\bibitem{MLMSWT10}
Maxwell, R.M., Lundquist, J.K., Mirocha, J.D., Smith, S.G., Woodward, C.S. and Tompson, A.F.B. Development of a coupled groundwater-atmospheric model. {\em Monthly Weather Review} doi:10.1175/2010MWR3392.

\bibitem{M10}
Maxwell, R.M. (2010). Infiltration in arid environments: Spatial patterns between subsurface heterogeneity and water-energy balances, {\em Vadose Zone Journal} 9, 970--983, doi:10.2136/vzj2010.0014.

\bibitem{M13}
Maxwell, R.M. (2013). A terrain-following grid transform and preconditioner for parallel, large-scale, integrated hydrologic modeling. {\em Advances in Water Resources} {\bf 53} 109--117, doi:10.1016/j.advwatres.2012.10.001. 


\bibitem{RMC10}
Rihani, J., Maxwell, R.M., Chow, F.K. (2010). Coupling groundwater and land-surface processes: Idealized simulations to identify effects of terrain and subsurface heterogeneity on land surface energy fluxes. {\em Water Resources Research} 46, W12523, doi:10.1029/2010WR009111.

\bibitem{SNSMM10}
Siirila, E.R., Navarre-Sitchler, A.K., Maxwell, R.M. and McCray, J.E. (2010). A quantitative methodology to assess the risks to human health from CO2 leakage into groundwater. {\em Advances in Water Resources} doi:10.1016/j.advwatres.2010.11.005.

\bibitem{SMPMPK10}
Sulis, M., Meyerhoff, S., Paniconi, C., Maxwell, R.M., Putti, M. and Kollet, S.J. (2010). A comparison of two physics-based numerical models for simulating surface water-groundwater interactions. { \em Advances in Water Resources}, 33(4), 456-467, doi:10.1016/j.advwatres.2010.01.010.

\bibitem{TAG89}
Tompson, A.F.B., R. Ababou, and L.W. Gelhar (1989).
\newblock Implementation of of the three-dimensional turning bands random field
  generator.
\newblock {\em Water Resources Research}, {\bf 25}(10):2227--2243.

\bibitem{TFSBA98}  Tompson, A.F.B., R.D. Falgout, S.G. Smith, W.J. Bosl and
S.F. Ashby (1998), Analysis of subsurface contaminant migration and
remediation using high performance computing, {\em Advances in Water
Resources}, { \bf 22}(3):203--221.

\bibitem{TBP99}
Tompson, A. F. B., C. J. Bruton, and G. A. Pawloski, eds. (1999b). {\em Evaluation of the hydrologic source term from underground nuclear tests in Frenchman Flat at the Nevada Test Site: The CAMBRIC test}, Lawrence Livermore National Laboratory, Livermore, CA (UCRL-ID-132300), 360pp. 

\bibitem{TCRM99}  Tompson, A.F.B., S.F. Carle, N.D. Rosenberg and R.M. Maxwell
 (1999). Analysis of groundwater migration from artificial recharge in a large
 urban aquifer: A simulation perspective, {\em Water Resources
Research}, {\bf 35}(10):2981--2998.

\bibitem{Teal02}
Tompson AFB., C.J. Bruton, G.A. Pawloski, D.K. Smith, W.L. Bourcier , D.E. Shumaker A.B. Kersting, S.F. Carle and R.M. Maxwell (2002). On the evaluation of groundwater contamination from underground nuclear tests.  {\em Environmental Geology}, {\bf 42}(2-3):235--247.

\bibitem{TMCZPS05}
Tompson, A. F. B., R. M. Maxwell, S. F. Carle, M. Zavarin, G. A. Pawloski, and D. E. Shumaker (2005). {\em Evaluation of the Non-Transient Hydrologic Source Term from the CAMBRIC Underground Nuclear Test in Frenchman Flat, Nevada Test Site}, Lawrence Livermore National Laboratory, Livermore, CA, UCRL-TR-217191.

\bibitem{VanGenuchten80}
{van Genuchten}, M.Th.(1980). 
\newblock A closed form equation for predicting the hydraulic conductivity of
  unsaturated soils.
\newblock {\em Soil Sci. Soc. Am. J.}, {\bf 44}:892--898.

\bibitem{welch.95}
B.~Welch (1995)
\newblock {\em Practical Programming in {T}{C}{L} and {T}{K}}.
\newblock Prentice Hall.

\bibitem{Woodward98}
Woodward, C.S. (1998),
\newblock A {N}ewton-{K}rylov-{M}ultigrid solver for variably saturated flow
  problems.
\newblock In {\em Proceedings of the XIIth International Conference on
  Computational Methods in Water Resources}, June.

\bibitem{WGM02}
Woodward, C.S., K.E. Grant, and R. Maxwell (2002). Applications of Sensitivity Analysis to Uncertainty Quantification for Variably Saturated Flow.
\newblock In {\em Proccedings of the XIVth International Conference on Computational Methods in Water Resources, Amsterdam}, The Netherlands, June.

\bibitem{endian}
{\em Endianness}, Wikipedia Entry: http://en.wikipedia.org/wiki/Endianness

\bibitem{Ajami14} 
Ajami, H., McCabe, M.F., Evans, J.P. and Stisen, S. (2014). Assessing the impact of model spin-up on surface water-groundwater interactions using an integrated hydrologic model. {\em Water Resources Research} {\bf 50} 2636�2656, doi:10.1002/2013WR014258.

\bibitem{AF96} 
Ashby, S. and Falgout, R. (1996). A parallel multigrid preconditioned conjugate gradient algorithm for groundwater flow simulations. {\em Nuclear Science and Engineering} {\bf 124} 145�159.

\bibitem{Atchley13a} 
Atchley, A.L., Maxwell, R.M. and Navarre-Sitchler, A.K. (2013). Human Health Risk Assessment of CO 2 Leakage into Overlying Aquifers Using a Stochastic, Geochemical Reactive Transport Approach. {\em Environmental Science and Technology} {\bf 47} 5954�5962, doi:10.1021/es400316c.

\bibitem{Atchley13b} 
Atchley, A.L., Maxwell, R.M. and Navarre-Sitchler, A.K. (2013). Using streamlines to simulate stochastic reactive transport in heterogeneous aquifers: Kinetic metal release and transport in CO2 impacted drinking water aquifers. {\em Advances in Water Resources} {\bf 52} 93�106, doi:10.1016/j.advwatres.2012.09.005.

\bibitem{Atchley11} 
Atchley, A.L. and Maxwell, R.M. (2011). Influences of subsurface heterogeneity and vegetation cover on soil moisture, surface temperature and evapotranspiration at hillslope scales. {\em Hydrogeology Journal} {\bf 19} 289�305, doi:10.1007/s10040-010-0690-1.

\bibitem{Burger12} 
B�rger, C.M., Kollet, S., Schumacher, J. and B�sel, D. (2012). Introduction of a web service for cloud computing with the integrated hydrologic simulation platform ParFlow. {\em Computers and Geosciences} {\bf 48} 334�336, doi:10.1016/j.cageo.2012.01.007.

\bibitem{Condon14a} 
Condon, L.E. and Maxwell, R.M. (2014). Feedbacks between managed irrigation and water availability: Diagnosing temporal and spatial patterns using an integrated hydrologic model. {\em Water Resources Research} {\bf 50} 2600�2616, doi:10.1002/2013WR014868.

\bibitem{Condon14b} 
Condon, L.E. and Maxwell, R.M. (2014). Groundwater-fed irrigation impacts spatially distributed temporal scaling behavior of the natural system: a spatio-temporal framework for understanding water management impacts. {\em Environmental Research Letters} {\bf 9} 1-9, doi:10.1088/1748-9326/9/3/034009.

\bibitem{Condon13a} 
Condon, L.E. and Maxwell, R.M. (2013). Implementation of a linear optimization water allocation algorithm into a fully integrated physical hydrology model. {\em Advances in Water Resources} {\bf 60} 135�147, doi:10.1016/j.advwatres.2013.07.012.

\bibitem{Condon13b} 
Condon, L.E., Maxwell, R.M. and Gangopadhyay, S. (2013). The impact of subsurface conceptualization on land energy fluxes. {\em Advances in Water Resources} {\bf 60} 188�203, doi:10.1016/j.advwatres.2013.08.001.

\bibitem{Cui14} 
Cui, Z., Welty, C. and Maxwell, R.M. (2014). Modeling nitrogen transport and transformation in aquifers using a particle-tracking approach. {\em Computers and Geosciences} doi:10.1016/j.cageo.2014.05.005.

\bibitem{Daniels11} 
Daniels, M.H., Maxwell, R.M. and Chow, F.K. (2011). Algorithm for Flow Direction Enforcement Using Subgrid-Scale Stream Location Data. {\em Journal of Hydrologic Engineering} {\bf 16} 677�683, doi:10.1061/(ASCE)HE.1943-5584.0000340.

\bibitem{deBarros09} 
de Barros, F.P.J., Rubin, Y. and Maxwell, R.M. (2009). The concept of comparative information yield curves and its application to risk-based site characterization. {\em Water Resources Research} {\bf 45} 1-16, doi:10.1029/2008WR007324.

\bibitem{deRooij13} 
de Rooij, R., Graham, W. and Maxwell, R.M. (2013). A particle-tracking scheme for simulating pathlines in coupled surface-subsurface flows. {\em Advances in Water Resources} {\bf 52} 7�18, doi:10.1016/j.advwatres.2012.07.022.

\bibitem{Ferg12} 
Ferguson, I.M. and Maxwell, R.M. (2012). Human impacts on terrestrial hydrology: climate change versus pumping and irrigation. {\em Environmental Research Letters} {\bf 7} 1-8, doi:10.1088/1748-9326/7/4/044022.

\bibitem{Ferg11} 
Ferguson, I.M. and Maxwell, R.M. (2011). Hydrologic and land�energy feedbacks of agricultural water management practices. {\em Environmental Research Letters} {\bf 6} 1-7, doi:10.1088/1748-9326/6/1/014006.

\bibitem{Ferg10} 
Ferguson, I.M. and Maxwell, R.M. (2010). Role of groundwater in watershed response and land surface feedbacks under climate change. {\em Water Resources Research} {\bf 46} 1-15, doi:10.1029/2009WR008616.

\bibitem{Frei09} 
Frei, S., Fleckenstein, J.H., Kollet, S.J. and Maxwell, R.M. (2009). Patterns and dynamics of river�aquifer exchange with variably-saturated flow using a fully-coupled model. {\em Journal of Hydrology} {\bf 375} 383�393, doi:10.1016/j.jhydrol.2009.06.038.

\bibitem{JW01} 
Jones, J.E. and Woodward, C.S. (2001). Newton�Krylov-multigrid solvers for large-scale, highly heterogeneous, variably saturated flow problems. {\em Advances in Water Resources} {\bf 24} 763�774, doi:10.1016/S0309-1708(00)00075-0.

\bibitem{Keyes13} 
Keyes, D.E., McInnes, L.C., Woodward, C., Gropp, W., Myra, E., Pernice, M., Bell, J., Brown, J., Clo, A., Connors, J., Constantinescu, E., Estep, D., Evans, K., Farhat, C., Hakim, A., Hammond, G., Hansen, G., Hill, J., Isaac, T., et al. (2013). Multiphysics simulations: Challenges and opportunities. {\em International Journal of High Performance Computing Applications} {\bf 27} 4�83, doi:10.1177/1094342012468181.

\bibitem{Kollat11} 
Kollat, J.B., Reed, P.M. and Maxwell, R.M. (2011). Many-objective groundwater monitoring network design using bias-aware ensemble Kalman filtering, evolutionary optimization, and visual analytics. {\em Water Resources Research} {\bf 47} 1-18, doi:10.1029/2010WR009194.

\bibitem{K10} 
Kollet, S.J., Maxwell, R.M., Woodward, C.S., Smith, S., Vanderborght, J., Vereecken, H. and Simmer, C. (2010). Proof of concept of regional scale hydrologic simulations at hydrologic resolution utilizing massively parallel computer resources. {\em Water Resources Research} {\bf 46} 1-7, doi:10.1029/2009WR008730.

\bibitem{K09a} 
Kollet, S.J. (2009). Influence of soil heterogeneity on evapotranspiration under shallow water table conditions: transient, stochastic simulations. {\em Environmental Research Letters} {\bf 4} 1-9, doi:10.1088/1748-9326/4/3/035007.

\bibitem{K09b} 
Kollet, S.J., Cvijanovic, I., Sch�ttemeyer, D., Maxwell, R.M., Moene, A.F. and Bayer, P. (2009). The Influence of Rain Sensible Heat and Subsurface Energy Transport on the Energy Balance at the Land Surface. {\em Vadose Zone Journal} {\bf 8} 846�857, doi:10.2136/vzj2009.0005.

\bibitem{K08a} 
Kollet, S.J. and Maxwell, R.M. (2008). Capturing the influence of groundwater dynamics on land surface processes using an integrated, distributed watershed model. {\em Water Resources Research} {\bf 44} 1-18, doi:10.1029/2007WR006004.

\bibitem{K08b} 
Kollet, S.J. and Maxwell, R.M. (2008). Demonstrating fractal scaling of baseflow residence time distributions using a fully-coupled groundwater and land surface model. {\em Geophysical Research Letters} {\bf 35} 1-6, doi:10.1029/2008GL033215.

\bibitem{K06} 
Kollet, S.J. and Maxwell, R.M. (2006). Integrated surface�groundwater flow modeling: A free-surface overland flow boundary condition in a parallel groundwater flow model. {\em Advances in Water Resources} {\bf 29} 945�958, doi:10.1016/j.advwatres.2005.08.006.
\bibitem{Major11} 
Major, E., Benson, D.A., Revielle, J., Ibrahim, H., Dean, A., Maxwell, R.M., Poeter, E. and Dogan, M. (2011). Comparison of Fickian and temporally nonlocal transport theories over many scales in an exhaustively sampled sandstone slab. {\em Water Resources Research} {\bf 47} 1-14, doi:10.1029/2011WR010857.

\bibitem{M14} 
Maxwell, R.M., Putti, M., Meyerhoff, S., Delfs, J.-O., Ferguson, I.M., Ivanov, V., Kim, J., Kolditz, O., Kollet, S.J., Kumar, M., Lopez, S., Niu, J., Paniconi, C., Park, Y.-J., Phanikumar, M.S., Shen, C., Sudicky, E. a. and Sulis, M. (2014). Surface-subsurface model intercomparison: A first set of benchmark results to diagnose integrated hydrology and feedbacks. {\em Water Resources Research} {\bf 50} 1531�1549, doi:10.1002/2013WR013725.

\bibitem{M13} 
Maxwell, R.M. (2013). A terrain-following grid transform and preconditioner for parallel , large-scale , integrated hydrologic modeling. {\em Advances in Water Resources} {\bf 53} 109�117, doi:10.1016/j.advwatres.2012.10.001.

\bibitem{M11} 
Maxwell, R.M., Lundquist, J.K., Mirocha, J.D., Smith, S.G., Woodward, C.S. and Tompson, A.F.B. (2011). Development of a Coupled Groundwater�Atmosphere Model. {\em Monthly Weather Review} {\bf 139} 96�116, doi:10.1175/2010MWR3392.1.

\bibitem{M10} 
Maxwell, R.M. (2010). Infiltration in Arid Environments: Spatial Patterns between Subsurface Heterogeneity and Water-Energy Balances. {\em Vadose Zone Journal} {\bf 9} 970�983, doi:10.2136/vzj2010.0014.

\bibitem{M09} 
Maxwell, R.M., Tompson, A.F.B. and Kollet, S. (2009). A serendipitous, long-term infiltration experiment: water and tritium circulation beneath the CAMBRIC trench at the Nevada Test Site. {\em Journal of contaminant hydrology} {\bf 108} 12�28, doi:10.1016/j.jconhyd.2009.05.002.

\bibitem{M08a} 
Maxwell, R.M. and Kollet, S.J. (2008). Quantifying the effects of three-dimensional subsurface heterogeneity on Hortonian runoff processes using a coupled numerical, stochastic approach. {\em Advances in Water Resources} {\bf 31} 807�817, doi:10.1016/j.advwatres.2008.01.020.

\bibitem{M08b} 
Maxwell, R.M. and Kollet, S.J. (2008). Interdependence of groundwater dynamics and land-energy feedbacks under climate change. {\em Nature Geoscience} {\bf 1} 665�669, doi:10.1038/ngeo315.

\bibitem{M08c} 
Maxwell, R.M., Carle, S.F. and Tompson, A.F.B. (2008). Contamination, risk, and heterogeneity: on the effectiveness of aquifer remediation. {\em Environmental Geology} {\bf 54} 1771�1786, doi:10.1007/s00254-007-0955-8.

\bibitem{M07a} 
Maxwell, R.M., Chow, F.K. and Kollet, S.J. (2007). The groundwater�land-surface�atmosphere connection: Soil moisture effects on the atmospheric boundary layer in fully-coupled simulations. {\em Advances in Water Resources} {\bf 30} 2447�2466, doi:10.1016/j.advwatres.2007.05.018.

\bibitem{M07b} 
Maxwell, R.M., Welty, C. and Harvey, R.W. (2007). Revisiting the Cape Cod Bacteria Injection Experiment Using a Stochastic Modeling Approach. {\em Environmental Science and Technology} {\bf 41} 5548�5558, doi: 10.1021/es062693a.

\bibitem{M05} 
Maxwell, R.M. and Miller, N.L. (2005). Development of a Coupled Land Surface and Groundwater Model. {\em Journal of Hydrometeorology} {\bf 6} 233�247, doi:10.1016/S0167-5648(04)80161-8.

\bibitem{Meyerhoff14a} 
Meyerhoff, S.B., Maxwell, R.M., Graham, W.D. and Williams, J.L. (2014). Improved hydrograph prediction through subsurface characterization: conditional stochastic hillslope simulations. {\em Hydrogeology Journal} doi:10.1007/s10040-014-1112-6.

\bibitem{Meyerhoff14b} 
Meyerhoff, S.B., Maxwell, R.M., Revil, A., Martin, J.B., Karaoulis, M. and Graham, W.D. (2014). Characterization of groundwater and surface water mixing in a semiconfined karst aquifer using time-lapse electrical resistivity tomography. {\em Water Resources Research} {\bf 50} 2566�2585, doi:10.1002/2013WR013991.

\bibitem{Meyerhoff11} 
Meyerhoff, S.B. and Maxwell, R.M. (2011). Quantifying the effects of subsurface heterogeneity on hillslope runoff using a stochastic approach. {\em Hydrogeology Journal} {\bf 19} 1515�1530, doi:10.1007/s10040-011-0753-y.

\bibitem{Mikkelson13} 
Mikkelson, K.M., Maxwell, R.M., Ferguson, I., Stednick, J.D., McCray, J.E. and Sharp, J.O. (2013). Mountain pine beetle infestation impacts: modeling water and energy budgets at the hill-slope scale. {\em Ecohydrology} {\bf 6} doi:10.1002/eco.278.

\bibitem{Rihani10} 
Rihani, J.F., Maxwell, R.M. and Chow, F.K. (2010). Coupling groundwater and land surface processes: Idealized simulations to identify effects of terrain and subsurface heterogeneity on land surface energy fluxes. {\em Water Resources Research} {\bf 46} 1-14, doi:10.1029/2010WR009111.

\bibitem{Shrestha14} 
Shrestha, P., Sulis, M., Masbou, M., Kollet, S. and Simmer, C. (2014). A scale-consistent Terrestrial Systems Modeling Platform based on COSMO, CLM and ParFlow. {\em Monthly Weather Review} doi:10.1175/MWR-D-14-00029.1.

\bibitem{Siirila12a} 
Siirila, E.R. and Maxwell, R.M. (2012). A new perspective on human health risk assessment: Development of a time dependent methodology and the effect of varying exposure durations. {\em Science of The Total Environment} {\bf 431} 221�232, doi:10.1016/j.scitotenv.2012.05.030.

\bibitem{Siirila12b} 
Siirila, E.R. and Maxwell, R.M. (2012). Evaluating effective reaction rates of kinetically driven solutes in large-scale, statistically anisotropic media: Human health risk implications. {\em Water Resources Research} {\bf 48} 1-23, doi:10.1029/2011WR011516.

\bibitem{Siirila12c} 
Siirila, E.R., Navarre-Sitchler, A.K., Maxwell, R.M. and McCray, J.E. (2012). A quantitative methodology to assess the risks to human health from CO2 leakage into groundwater. {\em Advances in Water Resources} {\bf 36} 146�164, doi:10.1016/j.advwatres.2010.11.005.

\bibitem{Sulis10} 
Sulis, M., Meyerhoff, S.B., Paniconi, C., Maxwell, R.M., Putti, M. and Kollet, S.J. (2010). A comparison of two physics-based numerical models for simulating surface water�groundwater interactions. {\em Advances in Water Resources} {\bf 33} 456�467, doi:10.1016/j.advwatres.2010.01.010.

\bibitem{Tompson 2002} 
Tompson, A., Bruton, C., Pawloski, G., Smith, D., Bourcier, W., Shumaker, D., Kersting, A., Carle, A. and Maxwell, R. (2002). On the evaluation of groundwater contamination from underground nuclear tests. {\em Environmental Geology} {\bf 42} 235�247, doi:10.1007/s00254-001-0493-8.

\bibitem{Tompson 1999} 
Tompson, A.F.B., Carle, S.F., Rosenberg, N.D. and Maxwell, R.M. (1999). Analysis of groundwater migration from artificial recharge in a large urban aquifer: A simulation perspective. {\em Water Resources Research} {\bf 35} 2981�2998, doi:10.1029/1999WR900175.

\bibitem{Williams11} 
Williams, J.L. and Maxwell, R.M. (2011). Propagating Subsurface Uncertainty to the Atmosphere Using Fully Coupled Stochastic Simulations. {\em Journal of Hydrometeorology} {\bf 12} 690�701, doi:10.1175/2011JHM1363.1.

\bibitem{Williams13} 
Williams, J.L., Maxwell, R.M. and Monache, L.D. (2013). Development and verification of a new wind speed forecasting system using an ensemble Kalman filter data assimilation technique in a fully coupled hydrologic and atmospheric model. {\em Journal of Advances in Modeling Earth Systems} {\bf 5} 785�800, doi:10.1002/jame.20051.


\end{thebibliography}
