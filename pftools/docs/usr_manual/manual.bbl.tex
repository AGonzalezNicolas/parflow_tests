\begin{thebibliography}{99}

\bibitem{Ashby-Falgout90}
Ashby, S.F. and R.~D. Falgout (1996).
\newblock A parallel multigrid preconditioned conjugate gradient algorithm for
  groundwater flow simulations.
\newblock {\em Nuclear Science and Engineering}, {\bf 124}:145--159.

\bibitem{Dai03}
Dai, Y., X. Zeng, R.E. Dickinson, I. Baker, G.B. Bonan, M.G. Bosilovich, A.S. Denning, P.A. Dirmeyer, P.R., G. Niu, K.W. Oleson, C.A. Schlosser and Z.L. Yang (2003). The common land model. {\em The Bulletin of the American Meteorological Society} {\bf 84}(8):1013--1023.

\bibitem{EW96}
Eisenstat, S.C. and H.F. Walker (1996).
\newblock Choosing the forcing terms in an inexact newton method.
\newblock {\em SIAM J. Sci. Comput.}, {\bf 17}(1):16--32.

\bibitem{Haverkamp-Vauclin81}
Haverkamp, R. and M.~Vauclin (1981).
\newblock A comparative study of three forms of the {R}ichard equation used for
  predicting one-dimensional infiltration in unsaturated soil.
\newblock {\em Soil Sci. Soc. of Am. J.}, {\bf 45}:13--20.


\bibitem{Jones-Woodward01}
Jones, J.E. and C.~S. Woodward (2001).
\newblock Newton-krylov-multigrid solvers for large-scale, highly heterogeneous, variably saturated flow problems.
\newblock {\em Advances in Water Resources}, {\bf 24}:763--774.

\bibitem{KM06}
Kollet, S.~J. and R.~M. Maxwell, (2006). Integrated
surface-groundwater flow
  modeling: A free-surface overland flow boundary condition in a parallel
  groundwater flow model. {\em Advances in Water Resources}, {\bf 29}:945--958 .

\bibitem{KM08a}
Kollet, S.J. and Maxwell, R.M. (2007). Capturing the influence of groundwater dynamics on land surface processes using an integrated, distributed watershed model, { \em Water Resources Research},{\bf 44}: W02402.

\bibitem{KM08b}
Kollet, S.J. and Maxwell, R.M. (2008). Demonstrating fractal scaling of baseflow residence time distributions using a fully-coupled groundwater and land surface model. {\em Geophysical Research Letters}, {\bf 35}, L07402. 

\bibitem{MWT03} Maxwell, R.M., C. Welty, and A.F.B. Tompson (2003).
Streamline-based simulation of virus transport resulting from long term
artificial recharge in a heterogeneous aquifer {\em Advances in Water
Resources}, {\bf 22}(3):203--221.

\bibitem{MM05}
Maxwell, R.M. and N.L. Miller (2005). Development of a coupled land surface and groundwater model.  {\em Journal of Hydrometeorology}, {\bf 6}(3):233--247.

\bibitem{MCT00}
Maxwell, R.M., S.F. Carle and A.F.B. Tompson (2000). Risk-Based Management of Contaminated Groundwater: The Role of Geologic Heterogeneity, Exposure and Cancer Risk in Determining the Performance of Aquifer Remediation, In {\em Proceedings of Computational Methods in Water Resources XII}, Balkema, 533--539.

\bibitem{MWH07} Maxwell, R.M., C. Welty and R.W. Harvey (2007). Revisiting the Cape Cod Bacteria Injection Experiment Using a Stochastic Modeling Approach, {\em Environmental Science and Technology}, { \bf 41}(15):5548--5558.

\bibitem{MCK07}
Maxwell, R.M., Chow, F.K. and Kollet, S.J., The groundwater-land-surface-atmosphere connection: soil moisture effects on the atmospheric boundary layer in fully-coupled simulations. {\em Advances in Water Resources}, {\bf 30}(12):2447--2466 .

\bibitem {MCT08}
Maxwell, R.M., Carle, S.F. and Tompson, A.F.B. (2008).
Contamination, Risk, and Heterogeneity: On the Effectiveness of Aquifer Remediation. {\em Environmental Geology}, {\bf 54}:1771--1786.

\bibitem{MK08a}
Maxwell, R.M. and Kollet, S.J. (2008). Quantifying the effects of three-dimensional subsurface heterogeneity on Hortonian runoff processes using a coupled numerical, stochastic approach. {\em Advances in Water Resources} {\bf 31}(5): 807--817. 

\bibitem{MK08b}
Maxwell, R.M. and Kollet, S.J. (2008) Interdependence of groundwater dynamics and land-energy feedbacks under climate change. {\em Nature Geoscience} {\bf 1}(10): 665--669.

\bibitem{TAG89}
Tompson, A.F.B., R. Ababou, and L.W. Gelhar (1989).
\newblock Implementation of of the three-dimensional turning bands random field
  generator.
\newblock {\em Water Resources Research}, {\bf 25}(10):2227--2243.

\bibitem{TFSBA98}  Tompson, A.F.B., R.D. Falgout, S.G. Smith, W.J. Bosl and
S.F. Ashby (1998), Analysis of subsurface contaminant migration and
remediation using high performance computing, {\em Advances in Water
Resources}, { \bf 22}(3):203--221.

\bibitem{TBP99}
Tompson, A. F. B., C. J. Bruton, and G. A. Pawloski, eds. (1999b). {\em Evaluation of the hydrologic source term from underground nuclear tests in Frenchman Flat at the Nevada Test Site: The CAMBRIC test}, Lawrence Livermore National Laboratory, Livermore, CA (UCRL-ID-132300), 360pp. 

\bibitem{TCRM99}  Tompson, A.F.B., S.F. Carle, N.D. Rosenberg and R.M. Maxwell
 (1999). Analysis of groundwater migration from artificial recharge in a large
 urban aquifer: A simulation perspective, {\em Water Resources
Research}, {\bf 35}(10):2981--2998.

\bibitem{Teal02}
Tompson AFB., C.J. Bruton, G.A. Pawloski, D.K. Smith, W.L. Bourcier , D.E. Shumaker A.B. Kersting, S.F. Carle and R.M. Maxwell (2002). On the evaluation of groundwater contamination from underground nuclear tests.  {\em Environmental Geology}, {\bf 42}(2-3):235--247.

\bibitem{TMCZPS05}
Tompson, A. F. B., R. M. Maxwell, S. F. Carle, M. Zavarin, G. A. Pawloski, and D. E. Shumaker (2005). {\em Evaluation of the Non-Transient Hydrologic Source Term from the CAMBRIC Underground Nuclear Test in Frenchman Flat, Nevada Test Site}, Lawrence Livermore National Laboratory, Livermore, CA, UCRL-TR-217191.

\bibitem{VanGenuchten80}
{van Genuchten}, M.Th.(1980). 
\newblock A closed form equation for predicting the hydraulic conductivity of
  unsaturated soils.
\newblock {\em Soil Sci. Soc. Am. J.}, {\bf 44}:892--898.

\bibitem{welch.95}
B.~Welch (1995)
\newblock {\em Practical Programming in {T}{C}{L} and {T}{K}}.
\newblock Prentice Hall.

\bibitem{Woodward98}
Woodward, C.S. (1998),
\newblock A {N}ewton-{K}rylov-{M}ultigrid solver for variably saturated flow
  problems.
\newblock In {\em Proceedings of the XIIth International Conference on
  Computational Methods in Water Resources}, June.

\bibitem{WGM02}
Woodward, C.S., K.E. Grant, and R. Maxwell (2002). Applications of Sensitivity Analysis to Uncertainty Quantification for Variably Saturated Flow.
\newblock In {\em Proccedings of the XIVth International Conference on Computational Methods in Water Resources, Amsterdam}, The Netherlands, June.

\end{thebibliography}
