%=============================================================================
% Chapter: Introduction
%=============================================================================

\chapter{Introduction}
\label{Introduction}

\parflow{} (\emph{PARallel FLOW}) is an integrated hydrology 
model that simulates surface and subsurface flow.
\parflow{} \cite{Ashby-Falgout90, Jones-Woodward01, KM06, M13} 
is a parallel simulation platform that operates in three modes:\begin{enumerate}
\item
steady-state saturated; 
\item
variably saturated; 
\item
and integrated-watershed flow.
\end{enumerate}
\parflow{} is especially suitable for large scale problems on a range
of single and multi-processor computing platforms. \parflow{}
simulates saturated and variably saturated
subsurface flow in heterogeneous porous media in three spatial
dimensions using a mulitgrid-preconditioned conjugate gradient solver
\cite{Ashby-Falgout90} and a Newton-Krylov nonlinear solver
\cite{Jones-Woodward01}. \parflow{} has recently been extended to
coupled surface-subsurface flow to enable the simulation of hillslope
runoff and channel routing in a truly integrated fashion
\cite{KM06}. \parflow{} is also fully-coupled with the land surface
model \code{CLM} \cite{Dai03} as described in \cite{MM05,KM08a}.  The
development and application of \parflow{} has been on-going for more
than 20 years \cite{Meyerhoff14a, Meyerhoff14b, Meyerhoff11, Mikkelson13,
Rihani10, Shrestha14, Siirila12a,
Siirila12b, Siirila12c, Sulis10, Williams11, Williams13, Ferg10, Keyes13, 
Kollat11, Condon13a, Condon13b, M13, KRM10, KRM10, SNSMM10, DMC10, AM10,
MLMSWT10, M10, FM10, KMWSVVS10, SMPMPK10, FFKM09, KCSMMB09, MTK09, dBRM08, 
MK08b, KM08b, KM08a, MK08a, MCT08,MCK07,MWH07,
  KM06, MM05, TMCZPS05, MWT03, Teal02, WGM02, Jones-Woodward01, MCT00,
  TCRM99, TBP99, TFSBA98, Ashby-Falgout90} and resulted in some of the
most advanced numerical solvers and multigrid preconditioners for
massively parallel computer environments that are available
today. Many of the numerical tools developed within the \parflow{}
platform have been turned into or are from libraries that are now
distributed and maintained at LLNL ({\em Hypre} and {\em SUNDIALS},
for example).  An additional advantage of \parflow{} is the use of a
sophisticated octree-space partitioning algorithm to depict complex
structures in three-space, such as topography, different hydrologic
facies, and watershed boundaries. All these components implemented
into \parflow{} enable large scale, high resolution watershed
simulations. 

\parflow{} is primarily written in \emph{C}, uses a modular
architecture and contains a flexible communications layer to
encapsulate parallel process interaction on a range of platforms.
\code{CLM} is fully-integrated into \parflow{} as a module and has
been parallelized (including I/O) and is written in \emph{FORTRAN
  90/95}.  \parflow{} is organized into a main executable
\file{\emph{pfdir}/pfsimulator/parflow_exe} and a library
\file{\emph{pfdir}/pfsimulator/parflow\_lib} (where \file{\emph{pfdir}} is
the main directory location) and is comprised of more than 190
separate source files.  \parflow{} is structured to allow it to be
called from within another application (\emph{e.g.} WRF, the Weather Research 
and Forecasting atmospheric model) or as a
stand-alone application.  There is also a directory structure for the
message-passing layer \file{\emph{pfdir}/pfsimulator/amps} for the
associated tools \file{\emph{pfdir}/pftools} for \code{CLM}
\file{\emph{pfdir}/pfsimulator/clm} and a directory of test cases
\file{\emph{pfdir}/test}.
This manual describes how to use \parflow{}, and is intended for
hydrologists, geoscientists, environmental scientists and engineers. 
This manual is written assuming the reader has a basic understanding
of Linux / UNIX environments, how to compose and execute scripts in various 
programming languages (e.g. TCL), and is familiar with groundwater and 
surface water hydrology, parallel computing, and numerical modeling in general.
In Chapter~\ref{Getting Started}, we describe how to install \parflow{}.
Then, we lead the user through a simple \parflow{} run.  In
Chapter~\ref{The ParFlow System}, we describe the \parflow{} system in
more detail.  Chapter~\ref{ParFlow Files} describes the formats of the
various files used by \parflow{}.  This manual provides some overview of \parflow{}
some information on building the code, examples of scripts that solve certain classes of
problems and a compendium of keys that are set for code options. 

\parflow{} has been used in a number of research studies published in the literature. 
What follows are tables of \parflow{} references with information on topics, types of problem and
application.  \ref{pfref1}, \ref{pfref2}, \ref{pfref3} and \ref{pfref4} describe the  \\
\newpage

{\tiny 
\begin{table}
\center
%\tabcolsep=.2cm
\caption{List of \parflow{} references with application and process details.}
\begin{tabular}{ p{1cm} p{2cm} l l p{1cm} p{1.5cm} p{1cm} p{1cm} | c | c | c | c | p{1cm} }
\bf{Author(s)} & \bf{Title} & \bf{Journal} & \bf{Year} & \bf{Coupled Model} & \bf{Application} & \bf{Scale} & \bf{Domain} & \bf{TB} & \bf{TFG} & \bf{VS} & \bf{Vdz} & \bf{Access/DOI} \\   
\hline{}
Cui et al. & Modeling Nitrogen Transport and Transformation in Aquifers Using a Particle-Tracking Approach & CAGEO & 2014 & SLIM-FAST & Model Development (nitrogen biogeochemistry) & Column to Hillslope & Idealized & X &   & X &  & DOI: 10.1016/j.cageo.2014.05.005  \\
Condon and Maxwell & Feedbacks between managed irrigation and water availability: Diagnosing temporal and spatial patterns using an integrated hydrologic model & WRR & 2014 & CLM & Agriculture & Watershed (1600 km2) & Little Washita, OK &  & X& X & & DOI: 10.1002/2013WR014868  \\
Meyerhoff et al. & Characterization of groundwater and surface water mixing in a semi-confined karst aquifer using time-lapse electrical resistivity tomography & WRR & 2014 & SLIM-FAST & Karst Environments & 2D & Transects in Santa Fe River Watershed & & X & & & DOI: 10.1002/2013WR013991  \\
Condon and Maxwell & Groundwater-fed irrigation impacts spatially distributed temporal scaling behavior of the natural system: A spatio-temporal framework for understanding water management impacts & ERL & 2014 & CLM & Agriculture & Watershed (1600 km2) & Little Washita, OK & & X & X & & DOI: 10.1016/j.advwatres.2013.07.012  \\
Maxwell et al. & Surface-subsurface model intercomparison: A first set of benchmark results to diagnose integrated hydrology and feedbacks & WRR & 2014 & Model Comparison & Many  & Idealized  &   &   & X  & X  &    \\ 
Meyerhoff et al. & Improved Hydrograph Prediction Through Subsurface Characterization: Conditional Stochastic Hillslope Simulations & HJ & 2014 &  & Stochastic runoff generation, conditioning & Hillslope & Idealized & X & &X & & DOI: 10.1007/s10040-014-1112-6  \\
Williams et al. & Development and verification of a new wind speed forecasting system using an Ensemble Kalman Filter data assimilation technique in a fully coupled hydrologic and atmospheric model & JAMES & 2013 & WRF & Atmosphere, DART, Data Assimilation & 15km × 15km × 5m & Idealized & X & & X & & DOI: 10.1002/jame.20051  \\
Condon et al. & The impact of subsurface conceptualization on land energy fluxes & AWR & 2013 & CLM & Subsurface Heterogeneity (land surface fluxes) & Watershed (56,000 km2) & Upper Klamath, OR & X & X & X & & DOI: 10.1016/j.advwatres.2013.08.001  \\
Condon and Maxwell & Implementation of a linear optimization water allocation algorithm into a fully integrated physical hydrology model & AWR & 2013 & CLM & Agriculture & Sub-Watershed (600 km2) & Little Washita, OK &  & X & X & & DOI: 10.1016/j.advwatres.2013.07.012   \\
\end{tabular}
\label{pfref1}
\end{table}

\begin{table} \center
\caption{List of \parflow{} references with application and process details (cont.).}

\begin{tabular}{ p{1cm} p{2cm} p{0.75cm} p{0.5cm} p{1cm} p{1.5cm} p{1cm} p{1cm} p{0.25cm} p{0.25cm} p{0.25cm} p{0.25cm} p{1cm} }
\bf{Author(s)} & \bf{Title} & \bf{Journal} & \bf{Year} & \bf{Coupled Model} & \bf{Application} & \bf{Scale} & \bf{Domain} & \bf{TB} & \bf{TFG} & \bf{VS} & \bf{Vdz} & \bf{Access/DOI} \\   
Atchley et al. & Human health risk assessment of CO2 leakage into overlying aquifers using a stochastic, geochemical reactive transport approach & ES\&T & 2013 & SLIM-FAST; CrunchFlow & Risk Assessment & 1200m × 300m × 60m & Idealized & Ÿ & DOI: 10.1021/es400316c   \\
Mikkelson et al. & Mountain pine beetle infestation impacts: Modeling water and energy budgets at the hill-slope scale & EH & 2013 & CLM & Mountain Pine Beetle & Hillslope (500m × 1000m × 12.5m) & Idealized & X & DOI: 10.1002/eco.278  \\
Keyes et al.  & Multiphysics simulations: Challenges and opportunities & IJHPCA & 2013 & Model Comparison & DOI: 10.1177/1094342012468181 \\
Maxwell & A terrain-following grid transform and preconditioner for parallel, large-scale, integrated hydrologic modeling & AWR & 2013 & Model Development & Idealized &  & DOI: 10.1016/j.advwatres.2012.10.001 \\
Atchley et al. & Using streamlines to simulate stochastic reactive transport in heterogeneous aquifers: Kinetic metal release and transport in CO2 impacted drinking water aquifers & AWR & 2013 & SLIM-FAST; CrunchFlow & Risk Assessment & 1200m × 300m × 60m & Idealized & Ÿ & DOI: 10.1016/j.advwatres.2012.09.005 \\ 
de Rooij et al. &  A particle-tracking scheme for simulating pathlines in coupled surface-subsurface flows & AWR & 2013 & SLIM-FAST & Model Development (surface particles) & Hillslope & Idealized &  & DOI: 10.1016/j.advwatres.2012.07.022  \\
Ferguson and Maxwell & Human impacts on terrestrial hydrology: Climate change versus pumping and irrigation & ERL & 2012 & CLM & Agriculture & Watershed (1600 km2) & Little Washita, OK &  & DOI: 10.1088/1748-9326/7/4/044022 \\
Siirila and Maxwell  & A new perspective on human health risk assessment: Development of a time dependent methodology and the effect of varying exposure durations & STE & 2012 & SLIM-FAST & Risk Assessment & 4000m × 1000m × 100m & Idealized & Ÿ & DOI: 10.1016/j.scitotenv.2012.05.030  \\
Siirila and Maxwell & Evaluating effective reaction rates of kinetically driven solutes in large-scale, statistically anisotropic media: Human health risk implications & WRR & 2012 & SLIM-FAST & Risk Assessment & 4000m × 1000m × 100m & Idealized & Ÿ & DOI: 10.1029/2011WR011516  \\
Siirila et al. & A quantitative methodology to assess the risks to human health from CO2 leakage into groundwater & AWR & 2012 & SLIM-FAST & Risk Assessment & 4000m × 1000m × 100m & Idealized & Ÿ & DOI: 10.1016/j.advwatres.2010.11.005  \\
Meyerhoff and Maxwell & Quantifying the effects of subsurface heterogeneity on hillslope runoff using a stochastic approach & HJ & 2011 & Subsurface Heterogeneity (runoff generation) & Hillslope (300m × 300m × 6m) & Idealized & Ÿ &  & Ÿ & DOI: 10.1007/s10040-011-0753-y  \\
Williams and Maxwell & Propagating subsurface uncertainty to the atmosphere using fully-coupled, stochastic simulations & JH & 2011 & WRF & Atmosphere & 15km × 15km × 5m & Idealized & Ÿ & Ÿ & DOI: 10.1175/2011JHM1363.1 \\
Ferguson and Maxwell & Hydrologic and land-energy feedbacks of agricultural water management practices & ERL & 2011 & CLM & Agriculture & Watershed (1600 km2) & Little Washita, OK & Ÿ & DOI: 10.1088/1748-9326/6/1/014006 \\
Daniels et al. & An algorithm for flow direction enforcement using subgrid-scale stream location data & JHE & 2011 & Streamflow & Owens Valley, CA floodplain & DOI: 10.1061/(ASCE)HE.1943-5584.0000340 \\
Atchley and Maxwell & Influences of subsurface heterogeneity and vegetation cover on soil moisture, surface temperature, and evapotranspiration at hillslope scales. & HJ & 2011 & CLM & Subsurface Heterogeneity (land surface processes) & Hillslope (250m × 250m × 2m) & Idealized & Ÿ & Ÿ & DOI: 10.1007/s10040-010-0690-1  \\
\end{tabular}
\label{pfref2}
\end{table}

\begin{table} \center
\caption{List of \parflow{} references with application and process details (cont.).}

\begin{tabular}{ p{1cm} p{2cm} p{0.75cm} p{0.5cm} p{1cm} p{1.5cm} p{1cm} p{1cm} p{0.25cm} p{0.25cm} p{0.25cm} p{0.25cm} p{1cm} }
\bf{Author(s)} & \bf{Title} & \bf{Journal} & \bf{Year} & \bf{Coupled Model} & \bf{Application} & \bf{Scale} & \bf{Domain} & \bf{TB} & \bf{TFG} & \bf{VS} & \bf{Vdz} & \bf{Access/DOI} \\   
Maxwell et al. & Development of a coupled groundwater-atmospheric model & MWR & 2011 & WRF & Atmosphere & Watershed (1600 km2) & Little Washita, OK & Ÿ & DOI: 10.1175/2010MWR3392.1 \\
Rihani et al. & Coupling groundwater and land-surface processes: Idealized simulations to identify effects of terrain and subsurface heterogeneity on land surface energy fluxes & WRR & 2010 & CLM & Subsurface Heterogeneity (land energy fluxes) & Hillslope (5000m × 100m × 310m/80m) & Idealized & Ÿ & DOI: 10.1029/2010WR009111  \\
Maxwell & Infiltration in arid environments: Spatial patterns between subsurface heterogeneity and water-energy balances & VZJ & 2010 & CLM & Subsurface Heterogeneity (infiltration) & Hillslope (1000m× 50m) & Idealized (Nevada Test Site) & Ÿ & Ÿ & DOI: 10.2136/vzj2010.0014  \\
Ferguson and Maxwell & The role of groundwater in watershed response and land surface feedbacks under climate change & WRR & 2010 & CLM & Agriculture & Watershed (1600 km2) & Little Washita, OK &  &  &  & DOI: 10.1029/2009WR008616 \\
Kollet et al. & Proof-of-concept of regional scale hydrologic simulations at hydrologic resolution utilizing massively parallel computer resources & WRR & 2010 & CLM & Computational Scaling & 45m x 45m & Idealized &  &  & Ÿ &  & DOI: 10.1029/2009WR008730 \\
Sulis et al. & A comparison of two physics-based numerical models for simulating surface water-groundwater interactions & AWR & 2010 & - & Model Comparison (CATHY) & 400m × 320m × 5m & Idealized & Ÿ & DOI: 10.1016/j.advwatres.2010.01.010 \\
Kollet et al. & The Influence of Rain Sensible Heat and Subsurface Energy Transport on the Energy Balance at the Land Surface & VZJ & 2009 & CLM & Heat Transport (ParFlowE) & Column & Idealized & Ÿ & DOI:10.2136/vzj2009.0005  \\
Frei et al. & Patterns and dynamics of river-aquifer exchange with variably-saturated flow using a fully-coupled model & JH & 2009 & - & Groundwater-Surface water exchange & 2 km by 5.5 km & Idealized river corridor & Ÿ & DOI: 10.1016/j.jhydrol.2009.06.038 \\
Maxwell and Kollet & Interdependence of groundwater dynamics and land-energy feedbacks under climate change & NG & 2008 & CLM & Climate Change (land-energy feedbacks to groudnwater) & Watershed (1600 km2) & Little Washita watershed & Ÿ & DOI: 10.1038/ngeo315 \\
Kollet and Maxwell & Demonstrating fractal scaling of baseflow residence time distributions using a fully-coupled groundwater and land surface model & GRL & 2008 & SLIM-FAST & Residence Time Distributions  & Watershed (1600 km2) & Little Washita watershed & Ÿ &  & DOI: 10.1029/2008GL033215  \\
Maxwell and Kollet & Quantifying the effects of three-dimensional subsurface heterogeneity on Hortonian runoff processes using a coupled numerical, stochastic approach & AWR & 2008 & - & Subsurface Heterogeneity (runoff) & Hillslope (3000  × 3000m) & Idealized & Ÿ & Ÿ &  & DOI: 10.1016/j.advwatres.2008.01.020 \\
Kollet and Maxwell & Capturing the influence of groundwater dynamics on land surface processes using an integrated, distributed watershed model & WRR & 2008 & CLM & Subsurface Heterogeneity (land energy fluxes) & Watershed (1600 km2) & Little Washita watershed & Ÿ &  & DOI: 10.1029/2007WR006004 \\
\end{tabular}
\label{pfref3}
\end{table}

\begin{table} \center
\caption{List of \parflow{} references with application and process details (cont.).}

\begin{tabular}{ p{1cm} p{2cm} p{0.75cm} p{0.5cm} p{1cm} p{1.5cm} p{1cm} p{1cm} p{0.25cm} p{0.25cm} p{0.25cm} p{0.25cm} p{1cm} }
\bf{Author(s)} & \bf{Title} & \bf{Journal} & \bf{Year} & \bf{Coupled Model} & \bf{Application} & \bf{Scale} & \bf{Domain} & \bf{TB} & \bf{TFG} & \bf{VS} & \bf{Vdz} & \bf{Access/DOI} \\   

Maxwell et al. & Revistiting the Cape Cod Bacteria Injection Experiment Using a Stochastic Modeling Approach & ES\&T & 2007 & particles & Subsurface Transport & 17m × 10.2m × 3.8m & Cape Cod & DOI: 10.1021/es062693a \\
Maxwell et al. & The groundwater-land-surface-atmosphere connection: soil moisture effects on the atmospheric boundary layer in fully-coupled simulations & AWR & 2007 & ARPS, CLM & Model Development (ARPS) & Watershed (1600 km2) & Little Washita watershed & Ÿ &  & DOI: 10.1016/j.advwatres.2007.05.018  \\
Kollet and Maxwell. & Integrated surface-groundwater flow modeling: A free-surface overland flow boundary condition in a parallel groundwater flow model & AWR & 2006 & - & Model Development (Overland Flow) \& Subsurface Heterogeneity (shallow overland flow) & Catchment (1.62 × 1km) & Idealized & Ÿ & Ÿ &  & DOI: 10.1016/j.advwatres.2005.08.006 \\
Maxwell and Miller & Development of a coupled land surface and groundwater model & JH & 2005 & CLM & Model Development (CLM) & Column & Ÿ &  & ? \\
Ajami et al. & Assessing the impact of model spin-up on surface water-groundwater interactions using an integrated hydrologic model & WRR & 2014 & CLM & Spin Up (initial conditions) & Watershed (208 km2) & Ringkobing Fjord subcatchment & Ÿ &  & DOI: 10.1002/2013WR014258 \\
Shrestha et al. & A scale-consistent Terrestrial Systems Modeling Platform based on COSMO, CLM and ParFlow & MWR & 2014 & COSMO-CLM & Model Development (TerrSysMP) & watershed (64 × 64 km and 150 × 150 km) & Idealized; Rur catchment & Ÿ &  & DOI: 10.1175/MWR-D-14-00029.1 \\
Burger et al. & Introduction of a web service for cloud computing with the integrated hydrologic simulation platform ParFlow & CG & 2012 & ParFlow Web & Model Development (ParFlow Web) &  & DOI:10.1016/j.cageo.2012.01.007 \\
Kollet & Influence of soil heterogeneity on evapotranspiration under shallow water table conditions: transient, stochastic simulations & ERL & 2009 & CLM & Subsurface Heterogeneity (evapotranspiration) & Column; 45m × 45m & Idealized & Ÿ &  & DOI:10.1088/1748-9326/4/3/035007 \\
\end{tabular}
\label{pfref4}
\end{table}

}
